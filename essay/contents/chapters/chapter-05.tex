\chapter{Conclusion: Contributions and Future Directions}

\section{Summary of Contributions: The Artifact and the Paradox}

The central investigation of this capstone project has been driven by a
critical examination of the tension between the industry's demand for agility
and the academic reality of architectural design methods. Through a systematic
mapping of the literature and an analysis of grey literature, this work
identified and defined the "Paradox of Software Architecture Decision-Making".
This paradox encapsulates the conflict where the industry trend towards
decentralized, autonomous teams—essential for maintaining a fast flow of
requirements—collides with the methodological reality that current
architectural derivation techniques "heavily rely on the tacit knowledge of
experienced practitioners". This reliance creates a bottleneck, effectively
reinforcing centralized decision-making structures and obstructing the
practical adoption of socio-technical architecture approaches.

The theoretical resolution to this paradox lies in ArchHypo, a technique that
shifts the architectural paradigm from static decision-making to dynamic
hypothesis engineering. By framing architectural uncertainties as testable
hypotheses, ArchHypo offers a structured pathway to manage risk and distribute
decision-making power. However, theoretical frameworks alone are insufficient
to change organizational behavior. Empirical evaluations of ArchHypo revealed
that while the technique is effective, it presents a steep learning curve and
administrative burden that can hinder widespread adoption. Consequently, this
project posits that the Paradox of Software Architecture Decision-Making
"cannot be resolved through process changes alone. It requires a technological
intervention".

To address this critical gap, the primary contribution of this capstone project
is the design and development of HypoStage, a specialized software artifact
designed to operationalize the ArchHypo framework. Recognizing that modern
engineering organizations are increasingly adopting Internal Developer Portals
(IDPs) to centralize their operations, HypoStage is architected as an "ArchHypo
plugin for Backstage". This strategic integration ensures that hypothesis
management is not an extraneous activity but is embedded directly into the
developer's daily workflow.

HypoStage serves as a "tangible solution to the identified gap in tool support
for managing architectural uncertainty". It translates the abstract concepts of
Hypothesis Engineering into a concrete digital interface, thereby reducing the
cognitive load required to adopt the methodology. By automating the
administrative aspects of the framework, the tool directly addresses the
documented need for better guidance to "lessen team dependency on architects".

The fundamental objective of HypoStage is to "operationalize the ArchHypo
framework". It achieves this by providing a structured environment that
"integrates architectural hypothesis management into your Backstage
environment, enabling teams to document, track, and validate architectural
decisions effectively". This operationalization transforms ArchHypo from a
theoretical concept into a pragmatic engineering practice.

Crucially, HypoStage acts as a mechanism for democratization. By embedding the
methodological rules, assessment criteria, and workflow steps directly into the
user interface, the tool provides the "structural scaffolding necessary for
less experienced practitioners to engage in decentralized, high-quality
architectural decision-making". This scaffolding externalizes the tacit
knowledge that was previously the exclusive domain of senior architects,
allowing broader team participation in the architectural process without
compromising integrity.

In terms of specific capabilities, HypoStage empowers engineering teams to
"Explicitly document and manage architectural hypotheses, assess them based on
impact and uncertainty, and define and track technical action plans". The tool
enforces rigor in the assessment process by utilizing "Likert scale ratings"
for evaluating Impact and Uncertainty, ensuring that qualitative assessments
are consistent and comparable across the organization. Through these features,
HypoStage provides the technological foundation necessary to break the cycle of
centralization and enable true socio-technical architectural evolution.

\section{Results: Insights Empowered by the Tool}

The development of HypoStage is grounded in the proven empirical benefits of
the ArchHypo technique. The tool is designed not merely to record data, but to
amplify and sustain the positive outcomes observed in industrial applications
of the methodology. By facilitating the management of architectural hypotheses,
the tool unlocks several critical advantages for software development projects.

\subsubsection{Support for Prioritization and Risk Management}

One of the most significant impacts of applying this methodology is the
enhancement of risk visibility. In complex projects, architectural risks are
often opaque or implicit. The structured elicitation and assessment of
hypotheses supported by the tool allow for "a more clear visualization of the
risks that could threaten the final delivery of the project". By making these
risks explicit and quantifiable through impact and uncertainty scores, teams
can prioritize their efforts more effectively, focusing on the most critical
uncertainties that could jeopardize project success.

\subsubsection{Sustainable Task Distribution and Planning}

The methodology provides a robust framework for managing the temporal dimension
of architecture. Rather than treating architecture as a monolithic upfront
phase, the approach supported by HypoStage enables a "structured approach to
dividing the architectural work through iterations". This allows for
"allocating and scheduling architectural tasks throughout project iterations",
ensuring that architectural evolution is continuous and integrated with feature
development. This distribution prevents the accumulation of unmanaged technical
debt and ensures that architectural work is prioritized alongside functional
requirements.

\subsubsection{Sustainable Development and Quality}

The systematic management of uncertainty contributes directly to the
sustainability of the development process. Empirical evidence suggests that the
adoption of this approach benefits projects by providing "predictability,
security, transparency, quality, and sustainable pace". Furthermore, the
technique has been shown to contribute significantly to "decision-making
processes and process efficiency". By reducing the chaos associated with
unforeseen architectural blockers, teams can maintain a steady velocity and
deliver higher-quality software that meets both functional and non-functional
requirements.

\subsubsection{Reduction of Upfront Design}

A core tenet of modern agile architecture is the avoidance of heavy upfront
design (BDUF), which often leads to waste and rigidity. The ArchHypo technique,
operationalized by the tool, allows projects to avoid "an upfront architectural
design". It achieves this by providing a safety net that supports the
"strategic postponement of decisions while addressing their potential impact".
By identifying which decisions can be safely delayed and monitoring their
associated risks, teams can maintain agility and keep their options open until
the "Most Responsible Moment" arises.

\subsubsection{Informed Decision-Making}

Finally, the ultimate goal of the tool is to elevate the quality of
architectural decisions. By moving away from intuition-based choices, the
methodology ensures that decisions are grounded in evidence and analysis. Teams
that have adopted this approach recognized that the management of hypotheses
"makes the team base the decisions on information". HypoStage facilitates this
by tracking the results of experiments, spikes, and analytics, ensuring that
when a decision is finally made, it is backed by data rather than guesswork.

\section{Future Steps: Validation and Research Evolution}

While the development of HypoStage represents a significant step towards
resolving the challenges of decentralized architectural decision-making, it
marks the beginning of a new phase of research and validation. The path forward
involves rigorous empirical testing of the tool, the refinement of the
underlying methodologies, and the continued expansion of the pattern language
that supports ArchHypo.

\subsubsection{Empirical Validation in Diverse Contexts}

To fully understand the efficacy and generalizability of HypoStage, it is
essential to conduct extensive empirical studies. Future work must "evaluate
the adoption of ArchHypo in companies and development teams with different
backgrounds and characteristics". This includes varying team sizes, domains,
and maturity levels. Furthermore, researchers must "apply these patterns in
other projects to study their broader impact, address challenges, and refine
solutions to improve their adoption and effectiveness". These studies should
aim to quantify the reduction in decision-making bottlenecks and the
improvement in architectural quality.

\subsubsection{User Acceptance Evaluation}

The success of any developer tool depends on its acceptance by the practitioner
community. To systematically evaluate how engineering teams perceive and
interact with HypoStage, future interview and survey studies should be designed
to identify the factors that influence the adoption of the tool—such as
performance expectancy, effort expectancy, and social influence. Gathering this
data will help refine both the user experience and the feature set.

\subsubsection{Tool Improvement and Learning Curve Reduction}

A primary motivation for HypoStage was to mitigate the steep learning curve
associated with ArchHypo. Consequently, continuous investigation into "tools
and methodologies to improve the implementation of ArchHypo represents an
important direction for research, potentially expanding its applicability" is
necessary. Future iterations of the tool should focus on features that provide
"more guidance on using the technique can be developed to reduce its learning
curve". This could include intelligent recommendation systems that suggest
potential hypotheses based on project characteristics or automated generation
of technical plans based on historical data.

\subsubsection{Evolution of the Framework via New Patterns}

The theoretical foundation of ArchHypo must also continue to evolve. Future
research should focus on "identifying and documenting patterns for various
types of hypotheses". As the software landscape changes, new categories of
uncertainty emerge. There is a specific need for the "exploration of common
uncertainties related to specific quality attributes, such as sustainability
and usability". Documenting these patterns will enrich the knowledge base
available to users of HypoStage, allowing them to leverage collective industry
knowledge when formulating their own hypotheses.

\subsubsection{Research on Process Guidelines}

Finally, the integration of ArchHypo into the broader software development
lifecycle remains a fertile area for research. The empirical study noted that
the introduction of specific process guidelines was a highly effective strategy
for managing uncertainty. Therefore, "future studies can also investigate how
these guidelines can be adopted as an approach to dealing with uncertainty".
Understanding how to best weave hypothesis engineering into Agile, DevOps, and
other process methodologies will be crucial for the seamless adoption of
decentralized architectural decision-making.
