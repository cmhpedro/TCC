\documentclass[a1paper,brazilian,english]{article}

\usepackage{imegoodies}
\usepackage[poster,hidelinks]{imelooks}
% \tcbposterset{fontsize = 32pt} % default, mude se necessário

% Diretórios onde estão as figuras; com isso, não é necessário (mas
% é permitido) colocar o caminho completo em \includegraphics. Note
% que a extensão nunca é necessária (mas é permitida), ou seja, o
% resultado é o mesmo com "\includegraphics{figuras/foto.jpeg}",
% "\includegraphics{foto.jpeg}", "\includegraphics{figuras/foto}"
% ou "\includegraphics{foto}".
\graphicspath{{figuras/},{fig/},{logos/},{img/},{images/},{imagens/}}

% Comandos rápidos para mudar de língua:
% \en -> muda para o inglês
% \br -> muda para o português
% \texten{blah} -> o texto "blah" é em inglês
% \textbr{blah} -> o texto "blah" é em português
\babeltags{br = brazilian, en = english}


%%%%%%%%%%%%%%%%%%%%%%%%%%%%%%%%%%%%%%%%%%%%%%%%%%%%%%%%%%%%%%%%%%%%%%%%%%%%%%%%
%%%%%%%%%%%%%%%%%%%%%%%%%%%%%%%%%% METADADOS %%%%%%%%%%%%%%%%%%%%%%%%%%%%%%%%%%%
%%%%%%%%%%%%%%%%%%%%%%%%%%%%%%%%%%%%%%%%%%%%%%%%%%%%%%%%%%%%%%%%%%%%%%%%%%%%%%%%

% O arquivo com os dados bibliográficos para biblatex; você pode usar
% este comando mais de uma vez para acrescentar múltiplos arquivos
\addbibresource{bibliografia.bib}

% Este comando permite acrescentar itens à lista de referências sem incluir
% uma referência de fato no texto (pode ser usado em qualquer lugar do texto)
%\nocite{bronevetsky02,schmidt03:MSc, FSF:GNU-GPL, CORBA:spec, MenaChalco08}
% Com este comando, todos os itens do arquivo .bib são incluídos na lista
% de referências
%\nocite{*}


%%%%%%%%%%%%%%%%%%%%%%%%%%%%%%%%%%%%%%%%%%%%%%%%%%%%%%%%%%%%%%%%%%%%%%%%%%%%%%%%
%%%%%%%%%%%%%%%%%%%%%%%%%%%%%%% INÍCIO DO POSTER %%%%%%%%%%%%%%%%%%%%%%%%%%%%%%%
%%%%%%%%%%%%%%%%%%%%%%%%%%%%%%%%%%%%%%%%%%%%%%%%%%%%%%%%%%%%%%%%%%%%%%%%%%%%%%%%


% Existem várias packages para criar pôsteres com LaTeX (a0poster, baposter,
% tikzposter, sciposter...). As mais comuns atualmente são beamerposter
% e tcolorbox (com sua biblioteca "poster"). Ambas funcionam muito bem;
% beamerposter é mais familiar (ela simplesmente utiliza beamer com alguns
% ajustes no tamanho das fontes e do papel), mas com tcolorbox o alinhamento
% vertical dos elementos é MUITO mais simples, e esta é a solução adotada
% aqui. Vale muito a pena ler a documentação com "texdoc tcolorbox" e
% "texdoc tcolorbox-tutorial-poster".

% Um pôster com tcolorbox é composto por blocos (posterboxes) coloridos
% de tamanho variável; cada bloco pode conter textos ou imagens e um
% título opcional. O pôster utiliza uma grade de dimensões definidas em
% \begin{tcposter} com "rows=" e "columns=" para fazer o alinhamento:
% para cada posterbox, podemos dizer "row=X, column=Y" para definir sua
% posição. Além disso, podemos dizer "span=A, rowspan=B" para fixar
% seu tamanho. Sem "span" e "rowspan", uma posterbox tem pelo menos o
% tamanho de uma célula da grade, mas se seu tamanho natural for maior
% ela extrapola esse tamanho. "span" e "rowspan" podem ser números
% não-inteiros (como 0.8 ou 1.4).
%
% "\begin{posterbox}" recebe um conjunto de parâmetros opcional e um
% conjunto de parâmetros obrigatório:
%
% "\begin{posterbox}[opcional]{obrigatório}".
%
% O conjunto de parâmetros opcional é onde inserimos os parâmetros comuns
% de tcolorbox, como "adjusted title", "coltext", "titlerule" etc.; o
% conjunto de parâmetros obrigatório é usado para determinar as dimensões
% e a posição da posterbox, ou seja, as opções "name", "column", "below",
% "span" etc.
%
% ALINHAMENTO HORIZONTAL
%
% É possível definir um poster com 2 colunas e fazer algo como
%
% \posterbox{column=1, span=1.3}{blah}
% \posterbox{column*=2, span=0.7}{blah}
%
% A segunda posterbox será alinhada à direita ("column*="), então as
% duas serão colocadas lado-a-lado sem sobreposições.
%
% Na prática, no entanto, é mais fácil fazer como no exemplo abaixo:
% definimos que o poster tem 12 colunas, o que nos permite dividir
% sua largura em 2, 3, 4 ou 6 colunas iguais ou diferentes (como
% 1/2 + 1/2, 2/3 + 1/3, 1/4 + 1/4 + 1/2, 1/4 + 1/6 + 1/4 + 1/3 etc).
%
% ALINHAMENTO VERTICAL
%
% Embora seja possível alinhar as posterboxes em função da grade na
% vertical, uma outra possibilidade é utilizar "above", "below" e
% "between", como no exemplo abaixo: basta associar um nome "blah" a
% uma determinada posterbox e, em outra, dizer "below=blah". Lembre-se
% que a posterbox de nome "blah" deve ser definida *antes* que outra
% possa fazer referência a ela. Também é possível fazer "below=top",
% "above=bottom" etc. A opção "equal height group" também é muito útil.
% Nada impede que você use estratégias de alinhamento diferentes para
% cada posterbox.

% Este modelo define a opção "smallmargins", que diminui a distância
% entre o conteúdo de uma posterbox e suas bordas. Use com parcimônia!

\begin{document}

\begin{tcbposter}[
		poster = {
				% showframe, % muito útil durante a preparação do poster
				rows = 6, columns = 12, colspacing = 1.2cm, rowspacing = .8cm, }, ]

	\posterbox[titlebox]{name=titlebox, below=top, column=1, span=12}{
		\begin{minipage}[c]{0.15\textwidth}
			\centering
			\includegraphics[width=\linewidth]{contents/images/ime.png}
		\end{minipage}
		\hfill
		\begin{minipage}[c]{0.80\textwidth}
			ArchHypo Tooling: Enabling Practical Hypothesis Engineering for SA\par
			\vspace{30pt}
			\normalsize\normalfont
			Author: Pedro Henrique Mariano Corrêa\par
			\vspace{4pt}
			Advisor: José Gonçalves Lima Neto |
			\vspace{4pt}
			Co-advisor: Paulo Meirelles
		\end{minipage}
	}

	\posterbox[adjusted title = Introduction, equal height group = introduction]
	{name=introduction, below=titlebox, column=1, span=6}{

		Software Architecture (SA) is the structural foundation essential for system
		sustainability and long-term viability. In modern, fast-paced environments, SA
		must be a dynamic, continuous process, moving beyond static blueprints. The
		core challenge is balancing development speed with quality to ensure
		sustainable and adaptable architectures. }

	\posterbox[adjusted title = HypoStage: Tool Overview, equal height group = introduction]
	{name=hypostage-overview, below=titlebox, column=7, span=6}{

		The \textbf{HypoStage} is a tool designed to help software teams manage
		architectural decisions more effectively in real-world projects. Built as a
		plugin for Backstage, it makes it easier for developers to document, assess,
		and track architectural hypotheses using structured guidance. It simplifies the
		process and reduces dependencies on experts. }

	\posterbox[adjusted title = The ArchHypo Framework]
	{name=archhypo, below=introduction, column=1, span=12}{

		\begin{itemize}
			\item \textbf{Hypothesis Engineering}: Uses hypotheses to manage and document architectural
			      uncertainties.
			      \vspace{12pt}
			\item \textbf{Uncertainty vs. Risk}: Focuses on unknowns, not just measurable risks.
			      \vspace{12pt}
			\item \textbf{Sources}: Uncertainties arise from requirements or technical solutions.
			      \vspace{12pt}
			\item \textbf{Assessment}: Hypotheses are rated by uncertainty level and impact.
			      \begin{itemize}
				      \vspace{12pt}
				      \item Uncertainty Level: How far the team is from proving the hypothesis true or
				            false. \vspace{12pt}
				      \item Impact: The effort/consequence required to transition to a different
				            alternative.
			      \end{itemize}
			      \vspace{12pt}
			\item \textbf{Technical Plan}: Plans aim to reduce key uncertainties or impacts.
		\end{itemize}
	}

	\posterbox[adjusted title = HypoStage: the ArchHypo tooling]
	{name=archhypo-tooling, below=archhypo, column=1, span=12}{

		\centering
		\includegraphics[width=0.49\textwidth]{contents/images/hypo-stage.png}
		\includegraphics[width=0.49\textwidth]{contents/images/hypo-stage-hypothesis.png}
	}

	\posterbox[adjusted title = HypoStage Functionalities]
	{name=hypostage, below=archhypo-tooling, column=1, span=12}{

		\begin{itemize}
			\item \textbf{Hypothesis Management}: Create, edit, and track hypothesis lifecycle.
			      \vspace{12pt}
			\item \textbf{Uncertainty Assessment}: Evaluate Impact and Uncertainty using structured 1--5 Likert scale.
			      \vspace{12pt}
			\item \textbf{Technical Planning}: Define and track technical action planslinked to hypotheses.
			      \vspace{12pt}
			\item \textbf{Quality Tracking}: Associate hypotheses with Quality Attributes.
			      \vspace{12pt}
			\item \textbf{Visualization \& Evidence}: Track evolution via interactive charts and link to supporting documentation.
		\end{itemize}
	}

	\posterbox[footerbox]{name=footerbox, above=bottom, column=1, span=12}{
		\centering
		More information: https://github.com/ArchHypo/hypo-stage\par
		\vspace{12pt}
		License: LGPL-3.0 }

\end{tcbposter}

\end{document}
