\chapter{Introduction}

\section{Software Architecture and the Challenge of Rapid Evolution}

In the modern landscape of software engineering, Software Architecture (SA) has
evolved from a static blueprint into a continuous, dynamic process that is
critical for business survival. As systems grow in complexity, SA serves as the
structural foundation that ensures long-term sustainability, acting as a
strategic asset that dictates an organization's agility in fluctuating markets.

Traditionally, architecture was defined by \cite{Perry1992} tripartite model
comprising elements (processing, data, and connections), form (relationships
and weights), and rationale (justification). However, the modern pressure to
deliver quickly often obfuscates the rationale, causing the form to degrade
into what is known as a "Big Ball of Mud".

The core tension today lies in balancing velocity with stability. While
architecture was once front-loaded, maintaining integrity while enabling rapid
feature delivery is now crucial, particularly in fast-paced environments like
startups where requirements emerge frequently. Practitioners must balance speed
with quality to ensure adaptable architectures; failing to do so results in
technical debt that paralyzes future development.

\section{The Central Paradox: Decentralization vs. Tacit Knowledge}

A profound dissonance exists between the industry's organizational aspirations
and the reality of architectural design methods. Modern trends, such as DevOps
and Team Topologies, advocate for democratizing decision-making to empower
autonomous teams and accelerate value flow.

However, this creates a "Paradox of Software Architecture Decision-Making".
While the industry pushes for decentralized, team-driven decisions, academic
evidence shows that existing methods for deriving architecture heavily rely on
the tacit knowledge of experienced practitioners. This reliance assumes the
presence of seasoned architects who can intuitively synthesize requirements—a
resource that is often scarce or expensive.

This dependency creates bottlenecks and limits the applicability of these
methods in teams lacking senior members, effectively concentrating power and
contradicting the ethos of distributed ownership. Consequently, a
self-reinforcing cycle emerges: organizations attempt to decentralize to handle
fast-flowing requirements, but are obstructed by frameworks that inherently
require centralization, blocking the adoption of a true socio-technical
approach.

\section{The Philosophical Solution: ArchHypo}

To resolve this paradox, this work builds upon the principles of ArchHypo, a
technique that shifts the view of architecture from a set of facts to a set of
experiments. ArchHypo employs hypotheses engineering to manage the
uncertainties inherent in software architecture.

By explicitly documenting uncertainties, ArchHypo allows for the strategic
postponement of decisions. This aligns with the "Continuous Architecture"
principle of delaying design decisions until the "Responsible Moment"—the point
where the cost of deciding is outweighed by the cost of waiting, and sufficient
information is available to minimize risk. Rather than making premature
commitments that lead to brittle systems, teams formulate technical plans based
on hypothesis assessment to mitigate impact and reduce uncertainty. This
evidence-based process allows decision-making to be safely distributed among
team members.

\section{The Practical Contribution: From Theory to Tooling}

While ArchHypo offers a robust theoretical framework, its practical application
faces significant friction due to the high cognitive load required to shift
from a feature-centric to a hypothesis-centric mindset. Observational studies
indicate that without guidance, teams find the technique hard to learn,
particularly regarding mapping risks and defining action plans.

There is a strong need for tool support to manage these hypotheses and execute
action plans to lessen dependency on senior architects. To address this, the
primary practical contribution of this capstone is the design and development
of HypoStage. As an ArchHypo plugin for Backstage, HypoStage integrates
directly into the developer's existing workflow within an Internal Developer
Portal (IDP).

HypoStage operationalizes ArchHypo by providing digital structures for
uncertainty assessment, quality attribute tracking, and technical planning. By
embedding these capabilities into a tool, this work aims to bridge the gap
between theoretical decentralization and the practical reality of fast-paced
development, effectively resolving the paradox of architectural
decision-making.
