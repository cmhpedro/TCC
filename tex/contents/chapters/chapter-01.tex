\chapter{Introduction}

Modern software engineering faces a significant conflict between the demand for
rapid feature delivery and the necessity of maintaining a stable, sustainable
Software Architecture (SA). While agile methodologies have successfully
introduced practices for functional evolution, they provide sparse guidance for
architectural evolution. This lack of recognized methodological support often
forces teams into a precarious choice: either performing heavy upfront designs
that contradict agile principles or allowing the system to degrade into a "Big
Ball of Mud" where architectural integrity is sacrificed for speed.

This methodological deficiency is compounded by a critical lack of operational
tooling to support architectural decision-making. As identified in the
systematic mapping by \cite{SysMap}, existing derivation methods rely heavily
on the tacit knowledge of experienced architects, offering little support for
less experienced practitioners or decentralized teams. Furthermore, research by
\cite{ArchHypo} emphasizes that even when robust frameworks like ArchHypo are
introduced to manage architectural uncertainty, their manual application
remains difficult and cognitively demanding. Consequently, a "tooling void"
exists where architectural management remains disconnected from the daily
engineering workflow, preventing teams from effectively bridging the gap
between high-level theory and the operational reality of continuous delivery.

\section{Software Architecture and the Challenge of Rapid Evolution}

In the modern landscape of software engineering, SA has evolved from a static
blueprint into a continuous, dynamic process that is critical for business
survival. As systems grow in complexity, SA serves as the structural foundation
that ensures long-term sustainability, acting as a strategic asset that
dictates an organization's agility in fluctuating markets.

Traditionally, architecture was defined by \cite{Perry1992} tripartite model
comprising elements (processing, data, and connections), form (relationships
and weights), and rationale (justification). However, the modern pressure to
deliver quickly often obfuscates the rationale, causing the form to degrade
into what is known as a "Big Ball of Mud", as described by \cite{BBOM}.

The core tension today lies in balancing velocity with stability. While
architecture was once front-loaded, maintaining integrity while enabling rapid
feature delivery is now crucial, particularly in fast-paced environments like
startups where requirements emerge frequently. Practitioners must balance speed
with quality to ensure adaptable architectures; failing to do so results in
technical debt that paralyzes future development.

\section{The Operational Gap: Agility Without Support}

While the software industry has embraced modern trends such as DevOps and Team
Topologies to empower autonomous teams and accelerate value flow, a significant
gap remains between these agile aspirations and the methodological support
available for architectural decision-making.

The industry increasingly demands that architecture evolve continuously
alongside code. However, existing methods for deriving and managing
architecture often lag behind, remaining abstract, manual, or disconnected from
the daily development workflow. There is a distinct lack of tooling that allows
teams to systematically manage architectural evolution without slowing down the
development pipeline. Without integrated support, teams struggle to apply
architectural rigor in real-time, often resorting to ad-hoc decisions that
bypass long-term structural considerations.

This deficiency creates a barrier to true agility. Organizations attempt to
decentralize ownership to handle fast-flowing requirements, but they lack the
concrete instruments to guide these decisions effectively. Consequently, the
evolution of software architecture becomes a friction point rather than an
enabler, highlighting the urgent need for frameworks and tools that bridge the
gap between high-level architectural theory and the operational reality of
continuous delivery.

\section{The Philosophical Solution: ArchHypo}

To address this operational challenge, this work builds upon the principles of
ArchHypo, a technique that shifts the view of architecture from a set of facts
to a set of experiments. ArchHypo employs hypotheses engineering to manage the
uncertainties inherent in software architecture.

By explicitly documenting uncertainties, ArchHypo allows for the strategic
postponement of decisions. This aligns with the "Continuous Architecture"
principle of delaying design decisions until the "Responsible Moment"—the point
where the cost of deciding is outweighed by the cost of waiting, and sufficient
information is available to minimize risk. Rather than making premature
commitments that lead to brittle systems, teams formulate technical plans based
on hypothesis assessment to mitigate impact and reduce uncertainty. This
evidence-based process allows decision-making to be safely distributed among
team members.

\section{The Practical Contribution: From Theory to Tooling}

While ArchHypo offers a robust theoretical framework, its practical application
faces significant friction due to the high cognitive load required to shift
from a feature-centric to a hypothesis-centric mindset. Observational studies
indicate that without guidance, teams find the technique hard to learn,
particularly regarding mapping risks and defining action plans.

There is a strong need for tool support to manage these hypotheses and execute
action plans, moving architectural management out of abstract documentation and
into the engineering lifecycle. To address this, the primary practical
contribution of this capstone is the design and development of HypoStage. As an
ArchHypo plugin for Backstage, HypoStage integrates directly into the
developer's existing workflow within an Internal Developer Portal (IDP).

HypoStage operationalizes ArchHypo by providing digital structures for
uncertainty assessment, quality attribute tracking, and technical planning. By
embedding these capabilities into a tool, this work aims to bridge the gap
between theoretical agility and the practical reality of fast-paced
development, providing the necessary infrastructure to support continuous
architectural evolution.
