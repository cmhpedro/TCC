\chapter{Introduction}

\section{Software Architecture and the Challenge of Rapid Evolution}

In the contemporary landscape of software engineering, the definition and
evolution of Software Architecture (SA) have transcended the traditional view
of static blueprints to become a dynamic and continuous process essential for
business survival. As software systems grow in complexity and scale, the
architectural foundation upon which they are built dictates not only their
current operational capacity but, more crucially, their future viability.
Consequently, "Software architecture (SA) is a fundamental concept in software
development, forming the structural foundation of systems and ensuring their
long-term sustainability." This sustainability is not merely a technical
concern; it is a strategic asset that determines an organization's agility in
response to market fluctuations.

Historically, the academic understanding of architecture has been grounded in
the tripartite definition proposed by Perry and Wolf, establishing SA as a
composition of "three critical components: elements, form, and rationale." The
\textit{elements} represent the processing, data, and connecting components;
the \textit{form} dictates the weighted properties and relationships among
them; and the \textit{rationale} provides the justification for these
structural choices. However, in the modern era of software
development—characterized by the proliferation of startups and the adoption of
Agile methodologies—the \textit{rationale} component faces unprecedented
pressure. The reasoning behind architectural decisions is often obfuscated by
the urgency of delivery, leading to systems where the "form" degrades into what
is colloquially known as a "Big Ball of Mud."

The core tension in modern software engineering lies in the temporal dimension
of design. In static environments, architecture could be heavily front-loaded;
however, "maintaining architectural integrity while enabling rapid feature
delivery is crucial in fast-paced environments such as software startups, where
requirements emerge frequently." This volatility in requirements necessitates a
paradigm shift from architecture as a fixed deliverable to architecture as a
continuous flow of decisions. The ability to pivot, scale, or refactor without
catastrophic system failure is the hallmark of a robust architecture.

Therefore, the overarching challenge for practitioners and researchers alike
resides in the delicate equilibrium between velocity and stability. The modern
challenge resides in balancing "speed with quality to ensure sustainable,
adaptable architectures that meet evolving business demands." Failure to
maintain this balance results in technical debt accumulation that eventually
paralyzes the development process, rendering the system incapable of evolving
alongside the business it supports.

\section{The Central Paradox: Decentralization vs. Tacit Knowledge}

Despite the widespread recognition of the need for architectural flexibility, a
profound dissonance exists between the organizational aspirations of the
software industry and the methodological realities of architectural design. The
industry, driven by the philosophy of DevOps and Team Topologies, increasingly
advocates for the democratization of decision-making. The goal is to empower
autonomous, cross-functional teams to own their architectural choices to
eliminate bottlenecks and accelerate the flow of value. However, this
aspiration creates a significant friction point when contrasted with the
prevailing methods for architectural derivation.

This study identifies and addresses the "Paradox of Software Architecture
Decision-Making," a phenomenon that "highlights the conflicting dynamics
between industry reports advocating for decentralized, team-driven
architectural decision-making and academic evidence showing that methods for
deriving software architecture still heavily rely on the tacit knowledge of
experienced practitioners."

This paradox is not merely anecdotal but is supported by rigorous secondary
studies. A systematic mapping of the literature has revealed that existing
methodologies for bridging the gap between requirements and architecture are
fundamentally flawed in their reliance on seniority. The mapping revealed that
existing methods "strongly relied on experienced practitioners' tacit knowledge
to derive the architectural definitions for a software system." This reliance
assumes the presence of a seasoned architect capable of intuitively
synthesizing complex requirements into coherent structures—a resource that is
scarce, expensive, and often unavailable in leaner startup environments.

The implications of this dependency are severe for organizations attempting to
scale. When architectural methods are opaque and rooted in the intuition of a
few experts rather than explicit, teachable processes, it "creates bottlenecks
and limits the practical applicability of these methods in teams without senior
members." It effectively concentrates power and responsibility in a centralized
manner, directly contradicting the agile ethos of distributed ownership.

Consequently, a vicious cycle emerges. Organizations strive for
decentralization to handle the fast flow of requirements, yet the tools and
methods available to them force a reversion to centralized command-and-control
structures to ensure architectural coherence. The result is a self-reinforcing
cycle where the change to decentralization is "obstructed by frameworks that
inherently require centralization, blocking the practical adoption of a
socio-technical architecture approach." Breaking this cycle requires a new
methodological approach that externalizes tacit knowledge and makes the
management of architectural uncertainty accessible to the broader engineering
team.

\section{The Philosophical Solution: ArchHypo}

To resolve the paradox of centralized knowledge in a decentralized world, this
work adopts and expands upon the principles of \textbf{ArchHypo}. This
technique represents a philosophical shift from viewing architecture as a set
of facts to viewing architecture as a set of experiments. By acknowledging that
early-stage architectural decisions are often based on assumptions rather than
certainties, ArchHypo "uses hypotheses engineering to manage uncertainties
related to software architecture."

The core philosophy of ArchHypo aligns seamlessly with the scientific method.
Rather than making premature commitments to specific technologies or patterns
based on incomplete information—a practice that leads to rigid and brittle
systems—ArchHypo advocates for the explicit documentation of uncertainties. The
objective is to allow the "strategic postponement of decisions while addressing
their potential impact." This is not procrastination; it is a deliberate risk
management strategy. By treating an architectural choice as a hypothesis (e.g.,
"The application should handle 1000 simultaneous requests"), the team can
define metrics and experiments to validate that choice before fully committing
to it.

This approach provides a structured mechanism for the implementation of the
"Continuous Architecture" principle to "delay design decisions." In traditional
agile environments, delaying a decision is often conflated with ignoring it.
ArchHypo, conversely, ensures that the delayed decision is actively managed. It
provides a framework that supports the team in "evaluating the respective
trade-offs in identifying the most responsible moment for that decision." The
"Responsible Moment" is the point in time where the cost of deciding is
outweighed by the cost of waiting, and where sufficient information has been
gathered to minimize risk.

Furthermore, ArchHypo moves beyond mere identification of uncertainty to
actionable engineering. It proposes "formulating a technical plan based on each
hypothesis' assessment, incorporating measures able to mitigate its impact and
reduce uncertainty." This might involve creating architectural spikes,
implementing tracer bullets, or designing abstraction layers that isolate the
uncertain component. Through this philosophical lens, architecture becomes an
iterative, evidence-based process that can be safely distributed among team
members, as the decision-making criteria are explicit rather than hidden within
an expert's intuition.

\section{The Practical Contribution: From Theory to Tooling}

While ArchHypo offers a robust theoretical framework for managing architectural
uncertainty, its practical application in real-world scenarios has encountered
significant friction. Theoretical elegance does not always translate to
operational efficiency without the appropriate support mechanisms. Empirical
research conducted to evaluate ArchHypo in industrial settings, although
demonstrating benefits such as a structured approach to architectural work,
revealed "significant adoption challenges."

The cognitive load required to shift from a feature-centric mindset to a
hypothesis-centric mindset is substantial. In observational studies, the
research team identified the "learning curve and process adjustments required
for ArchHypo's adoption as significant challenges that could hinder its
widespread adoption." Engineering teams, often under pressure to deliver
functional requirements, found the manual management of hypotheses to be
administratively burdensome. Specifically, participants found the technique
"hard to learn, particularly in mapping risks, specifying hypotheses, and
defining action plans." Without guidance, the specification of hypotheses was
inconsistent, and the link between a hypothesis and its technical mitigation
plan was often tenuous.

These findings underscored a critical gap in the current state of the art:
there is "a strong need for better guidance and, crucially, tool support to
manage hypotheses and execute action plans, which could lessen team dependency
on architects." If the goal is to democratize architectural decision-making,
the method must be supported by tooling that guides less experienced developers
through the process, reducing the reliance on the tacit knowledge of senior
architects.

Therefore, the primary practical contribution of this capstone is the design
and development of \textbf{HypoStage}. Recognizing the industry trend towards
Internal Developer Portals (IDPs) as central hubs for engineering operations,
HypoStage is defined as an "ArchHypo plugin for Backstage." By integrating
directly into the developer's existing workflow within the Backstage ecosystem,
HypoStage aims to lower the barrier to entry for Hypothesis Engineering.

HypoStage provides a comprehensive digital structure that operationalizes the
theoretical concepts of ArchHypo. It includes features for "Uncertainty
Assessment, Quality Attributes Tracking, and Technical Planning capabilities,"
allowing teams to document hypotheses, score them based on impact and
uncertainty, and track the execution of experimental mitigation plans. Through
this tool, this work aims to bridge the gap between the theoretical promise of
decentralized decision-making and the practical reality of fast-paced software
development, finally resolving the paradox of architectural decision-making.
