\chapter{The HypoStage Tooling Solution}

This chapter presents the practical contribution of this research: the
conception, architectural design, and implementation of \textbf{HypoStage}.
Based on the theoretical foundations established in the previous chapter and
the empirical gaps identified in the literature regarding the ArchHypo
framework, this chapter details the development of a software tool designed to
operationalize decentralized architectural decision-making. The chapter begins
by justifying the necessity of a technological intervention to resolve the
operational gap between architectural intent and execution. It then proceeds to
describe the solution's integration within the Backstage internal developer
portal, details the functional requirements for managing the hypothesis
lifecycle, and concludes with a technical examination of the software
architecture and the open-source strategy adopted for the project.

\section{Justification for Tooling Support}

The theoretical framework of ArchHypo, as discussed in Chapter 2, provides a
robust methodology for managing architectural uncertainty through Hypothesis
Engineering. However, the transition from theory to practice presents
significant hurdles. Empirical research conducted by \cite{ArchHypo}. on the
application of ArchHypo has demonstrated clear benefits, such as improved
collaboration and a structured approach to architectural work, yet it
simultaneously revealed significant adoption challenges.

\subsection{The Dilemma of Manual Application}

The primary barrier to the widespread adoption of Hypothesis Engineering in
software architecture lies in the cognitive load and process overhead imposed
on development teams. Participants in empirical studies found the technique
hard to learn, particularly in mapping risks, specifying hypotheses, and
defining action plans. Specifically, the abstract nature of translating vague
architectural uncertainties into testable hypotheses proved to be a cognitive
bottleneck. The study highlighted that teams struggled with mapping the
scenarios with risk and properly articulating the necessary technical
interventions.

Furthermore, without structured guidance, the execution of the framework often
defaulted back to the most experienced members of the team, reinforcing the
very centralization the method aims to dismantle. The feedback from
practitioners was explicit regarding the solution to this problem: there is a
strong need for tool support to manage hypotheses and execute action plans,
which could lessen team dependency on architects. Without effective tools to
aid architects—suggesting patterns and supporting decision-making—the
dependency on seasoned experts remains a bottleneck, perpetuating the
centralization of architectural authority.

\subsection{Technological Intervention as a Scaffolding Mechanism}

This capstone project posits that the dissonance identified in Chapter 2 —
where the desire for decentralized autonomy conflicts with the reliance on
centralized tacit knowledge—cannot be resolved through process changes alone.
It requires a technological intervention. The development of a dedicated
software tool aims to operationalize the ArchHypo framework.

By embedding the methodological rules, assessment scales, and technical
patterns directly into a software interface, the tool provides the necessary
structural scaffolding for less experienced practitioners. It intends to allow
teams to explicitly document and manage architectural hypotheses, assess them
based on impact and uncertainty, and define and track technical action plans.
Consequently, the tool acts as a mechanism for knowledge distribution, enabling
the decentralization of high-quality architectural decision-making.

\section{HypoStage and the Backstage Platform Strategy}

To address the identified needs, this project introduces \textbf{HypoStage}, a
software solution designed to integrate architectural hypothesis management
into the daily workflow of engineering teams.

\subsection{The HypoStage Solution}

HypoStage is defined as an ArchHypo plugin for Backstage. It is not merely a
documentation repository but an active management tool that integrates
architectural hypothesis management into the organization's IDP, enabling teams
to document, track, and validate architectural decisions effectively. The tool
is designed to guide users through the lifecycle of a hypothesis, from
elicitation to validation or refutation, ensuring that uncertainty is treated
as a first-class citizen in the software delivery process.

A key design principle of HypoStage is its capability to seamlessly integrate
with Backstage's catalog and entity system. This ensures that architectural
hypotheses are not abstract entities but are directly linked to the software
components, APIs, and resources they affect, promoting traceability and
context-aware decision-making.

\subsection{Rationale for the Backstage Ecosystem}

The decision to implement HypoStage as a plugin for \textbf{Backstage}—an
open-source IDP—is strategic. To bridge the gap identified in the literature,
researchers have suggested that plugins of existing tools could introduce
support for hypotheses management. By integrating with project management
platforms or developer portals, a tool can facilitate seamless adoption into
diverse development workflows.

Backstage was selected as the host platform because it serves as a centralized
hub for software development teams, aggregating infrastructure, services, and
documentation. By placing hypothesis management within the IDP, HypoStage
ensures that architectural decision-making occurs in the same environment where
development happens. This visibility is crucial for decentralization, as it
democratizes access to architectural rationale.

\section{Functional Requirements and Implementation}

HypoStage implements specific functional requirements derived from the ArchHypo
framework to support the complete lifecycle of hypothesis engineering. These
functionalities cover documentation, assessment, planning, and visualization.

\subsection{Explicit Hypothesis Management and Tracking}

The core functionality of HypoStage is Hypothesis Management, which allows
users to create, edit, and track architectural hypotheses with detailed
metadata. The system enforces a structured format for elicitation, requiring
clear statements that define the uncertainty. The implementation provides a
robust form interface that captures essential data points such as the
Hypothesis Statement, Source Type (e.g., Requirements, Solution), and
associated Entity References from the catalog.

The process of creating a hypothesis in HypoStage follows a step-by-step
workflow, as illustrated in Figures~\ref{fig:create-hypothesis-1},
\ref{fig:create-hypothesis-2}, and \ref{fig:create-hypothesis-3}. Users begin
by accessing the hypothesis creation form, where they can input the hypothesis
statement and select relevant metadata. The interface guides users through each
step, ensuring that all necessary information is captured before the hypothesis
is created.

\begin{figure}
	\centering
	\includegraphics[width=0.9\textwidth]{contents/images/create-hypothesis-1}
	\caption{Step 1: Accessing the hypothesis creation form in HypoStage}
	\label{fig:create-hypothesis-1}
\end{figure}

\begin{figure}
	\centering
	\includegraphics[width=0.9\textwidth]{contents/images/create-hypothesis-2}
	\caption{Step 2: Filling in the hypothesis details and metadata}
	\label{fig:create-hypothesis-2}
\end{figure}

\begin{figure}
	\centering
	\includegraphics[width=0.9\textwidth]{contents/images/create-hypothesis-3}
	\caption{Step 3: Completing the hypothesis creation process}
	\label{fig:create-hypothesis-3}
\end{figure}

Furthermore, the tool supports Status Tracking, allowing teams to monitor
hypothesis lifecycle from creation to validation. The system supports multiple
statuses for hypotheses, such as Open, In Review, Validated, Discarded, and
Trigger-Fired. This explicit tracking ensures that architectural uncertainties
are not forgotten but are actively managed until resolution.

Once hypotheses are created, users can view them in a comprehensive list view,
as shown in Figure~\ref{fig:hypothesis-list}, which provides an overview of all
hypotheses and their current status. Clicking on a specific hypothesis opens
its detailed page, illustrated in Figure~\ref{fig:hypothesis-page}, where users
can view complete information, edit details, and track the hypothesis
lifecycle.

\begin{figure}
	\centering
	\includegraphics[width=0.9\textwidth]{contents/images/hypothesis-list}
	\caption{The list view showing all created hypotheses in HypoStage}
	\label{fig:hypothesis-list}
\end{figure}

\begin{figure}
	\centering
	\includegraphics[width=0.9\textwidth]{contents/images/hypothesis-page}
	\caption{A detailed view of a hypothesis page in HypoStage}
	\label{fig:hypothesis-page}
\end{figure}

\subsection{Uncertainty and Impact Assessment}

A critical component of the ArchHypo framework is the quantitative and
qualitative assessment of hypotheses. HypoStage implements Uncertainty
Assessment functionality to evaluate hypothesis uncertainty using Likert scale
ratings.

The interface requires users to provide an Uncertainty Rating and an Impact
Rating, both utilizing a 1-5 scale. These ratings map to values ranging from
Very Low (1) to Very High (5). This structured assessment forces teams to
critically evaluate how far they are from validating a hypothesis and what the
potential consequences of failure are, directly addressing the difficulty
participants faced in mapping risks in manual implementations.

\subsection{Technical Planning and Quality Correlation}

To move from assessment to action, HypoStage provides Technical Planning
capabilities. The tool allows teams to create and manage technical planning
items linked to hypotheses. Users can define specific actions—such as
Architectural Spike, Tracer Bullet, or Prototype — assigned to specific
entities with target dates and expected outcomes. This directly supports the
objective to define and track technical action plans associated with each
hypothesis.

The creation of a technical plan follows a structured process, as demonstrated
in Figures~\ref{fig:create-tech-plan-1} and \ref{fig:create-tech-plan-2}. Users
can initiate the creation of a technical plan from within a hypothesis page,
where they are guided through defining the action type, target entities, and
expected outcomes. Once created, technical plans are displayed within the
hypothesis page, as shown in Figure~\ref{fig:tech-plan-list}, providing
visibility into all planned actions associated with a given hypothesis.

\begin{figure}
	\centering
	\includegraphics[width=0.9\textwidth]{contents/images/create-tech-plan-1}
	\caption{Step 1: Initiating the creation of a technical plan in HypoStage}
	\label{fig:create-tech-plan-1}
\end{figure}

\begin{figure}
	\centering
	\includegraphics[width=0.9\textwidth]{contents/images/create-tech-plan-2}
	\caption{Step 2: Defining the technical plan details and action items}
	\label{fig:create-tech-plan-2}
\end{figure}

\begin{figure}
	\centering
	\includegraphics[width=0.9\textwidth]{contents/images/tech-plan-list}
	\caption{Technical plans displayed within a hypothesis page after creation}
	\label{fig:tech-plan-list}
\end{figure}

Additionally, the tool implements Quality Attributes Tracking, enabling users
to associate hypotheses with specific quality attributes. The system supports a
comprehensive list of attributes, such as Performance, Security, Scalability,
and Maintainability. This feature ensures that architectural decisions are
explicitly linked to the non-functional requirements they impact, fostering a
quality-driven architectural culture.

\subsection{Visualization and Evidentiary Support}

To support decision-making over time, HypoStage includes Visualization features
to track hypothesis evolution and validation status through interactive charts.
The system renders a temporal view of how uncertainty and impact ratings change
as technical plans are executed, providing empirical evidence of risk
reduction.

Finally, to ground decisions in facts rather than intuition, the tool supports
the inclusion of Evidence URLs. The system allows teams to add supporting
documentation links, such as external test results, POC repositories, or vendor
documentation, thereby ensuring that the validation of a hypothesis is
auditable and transparent.

\section{Software Engineering Architecture}

The engineering of HypoStage follows modern software development practices,
utilizing a decoupled architecture that aligns with the Backstage plugin
ecosystem. The solution is divided into two main packages: the frontend plugin
and the backend plugin.

\subsection{Component Architecture}

The system architecture is composed of distinct frontend components and backend
services.

\subsubsection{Frontend Architecture}

The frontend is built using React and integrates with the Backstage core API.

\subsubsection{Backend Architecture}

The backend logic is encapsulated in the backend package.

\begin{itemize}
	\item HypothesisService: The core service for hypothesis management. As seen in this
	      service manages business logic and data persistence. It utilizes database
	      transactions to ensure data integrity when creating hypotheses and logging
	      associated lifecycle events simultaneously.
	\item Router: The router module uses express-promise-router to define RESTful
	      endpoints, handling request validation via Zod schemas before invoking the
	      service layer.
	\item Persistence: The system uses Knex.js for database interactions, including
	      database migrations for schema setup which define the structure for hypothesis,
	      technicalPlanning, and hypothesisEvents tables.
\end{itemize}

\subsection{Design for Integration and Modularity}

The project adheres to the goal of ensuring modularity and integrability. The
plugin is registered within the Backstage backend system using the
createBackendPlugin factory, allowing it to inject dependencies such as logger,
database, httpAuth, and catalogService.

This dependency injection model is crucial for the tool's operation as a
socio-technical enabler. For instance, the HypothesisService interacts with the
CatalogService to fetch entity references, ensuring that hypotheses are tightly
coupled to the actual software components registered in the organization's
ecosystem. This design allows HypoStage to be used as a standalone application
(in a dev environment) or integrated with existing developer portals,
fulfilling the requirement for seamless adoption.

\section{Open Source Commitment and Documentation}

Recognizing that the challenges of architectural decision-making are universal,
HypoStage is positioned as a contribution to the wider software engineering
community.

\subsection{Licensing and Distribution}

The project is distributed as open-source software under the LGPL-3.0 license
(GNU Lesser General Public License v3.0). This licensing model was chosen to
balance the freedom of use with the requirement that modifications to the
plugin library itself remain open source, thereby encouraging community
contribution back to the core tool while allowing integration into proprietary
Backstage instances.

\subsection{Documentation for Adoption}

To mitigate the learning curve not just of the method, but of the tool itself,
comprehensive documentation has been developed. The README.md serves as the
primary entry point, detailing Installation, Configuration, and Usage guides.
It explains how to configure the frontend routes and backend plugins, ensuring
that organizations can adopt the tool with minimal friction. By providing clear
API Reference and Features descriptions, the project aims to lower the barrier
to entry, facilitating the shift from centralized, tacit architectural
management to a transparent, decentralized, and hypothesis-driven approach.
