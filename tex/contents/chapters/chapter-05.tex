\chapter{Conclusion: Contributions and Future Directions}

\section{Summary of Contributions: Filling the Tooling Void}

This capstone project addresses a fundamental operational void in modern
software architecture: the critical absence of specialized tooling to manage
architectural uncertainty. While the theoretical foundations for decentralized
decision-making exist in frameworks like ArchHypo, their practical application
has been stifled by a reliance on manual, high-friction processes that
development teams struggle to sustain.

The primary contribution of this work is the delivery of the missing
technological link: **HypoStage**.

HypoStage is a purpose-built software artifact designed to operationalize
Hypothesis Engineering. It transforms the abstract principles of architectural
uncertainty—previously managed through ad-hoc documentation or the tacit
knowledge of senior architects—into a concrete, structured digital workflow. By
automating the lifecycle of hypothesis elicitation, assessment, and technical
planning, the tool removes the administrative burden that has historically
blocked the adoption of evidence-based architectural methods.

Recognizing that modern engineering organizations are consolidating their
operations into Internal Developer Portals (IDPs), HypoStage was not built in
isolation but as a fully integrated plugin for **Backstage**. This strategic
architectural choice ensures that hypothesis management is embedded directly
into the developer's "Golden Path." By placing the tool where developers
already manage their services and deployments, HypoStage bridges the gap
between architectural intent and daily engineering execution, providing the
necessary scaffolding for teams to practice continuous, decentralized
architecture effectively.

\section{Results: Insights Empowered by the Tool}

The development of HypoStage is grounded in the proven empirical benefits of
the ArchHypo technique. The tool is designed not merely to record data, but to
amplify and sustain the positive outcomes observed in industrial applications
of the methodology. By facilitating the management of architectural hypotheses,
the tool unlocks several critical advantages for software development projects.

\subsubsection{Support for Prioritization and Risk Management}

One of the most significant impacts of applying this methodology is the
enhancement of risk visibility. In complex projects, architectural risks are
often opaque or implicit. The structured elicitation and assessment of
hypotheses supported by the tool allow for a clearer visualization of the risks
that could threaten the final delivery of the project.

By making these risks explicit and quantifiable through impact and uncertainty
scores, teams can prioritize their efforts more effectively, focusing on the
most critical uncertainties that could jeopardize project success.

\subsubsection{Sustainable Task Distribution and Planning}

The methodology provides a robust framework for managing the temporal dimension
of architecture. Rather than treating architecture as a monolithic upfront
phase, the approach supported by HypoStage enables a structured approach to
dividing the architectural work through iterations. This allows for allocating
and scheduling architectural tasks throughout project iterations, ensuring that
architectural evolution is continuous and integrated with feature development.
This distribution prevents the accumulation of unmanaged technical debt and
ensures that architectural work is prioritized alongside functional
requirements.

\subsubsection{Sustainable Development and Quality}

The systematic management of uncertainty contributes directly to the
sustainability of the development process. Empirical evidence suggests that the
adoption of this approach benefits projects by providing predictability,
security, transparency, and a sustainable pace. Furthermore, the technique has
been shown to contribute significantly to decision-making efficiency. By
reducing the chaos associated with unforeseen architectural blockers, teams can
maintain a steady velocity and deliver higher-quality software that meets both
functional and non-functional requirements.

\subsubsection{Reduction of Upfront Design}

A core tenet of modern agile architecture is the avoidance of heavy upfront
design, which often leads to waste and rigidity. The ArchHypo technique,
operationalized by the tool, allows projects to avoid an upfront architectural
design. It achieves this by providing a safety net that supports the strategic
postponement of decisions while addressing their potential impact. By
identifying which decisions can be safely delayed and monitoring their
associated risks, teams can maintain agility and keep their options open until
the "Most Responsible Moment" arises.

\subsubsection{Informed Decision-Making}

Finally, the ultimate goal of the tool is to elevate the quality of
architectural decisions. By moving away from intuition-based choices, the
methodology ensures that decisions are grounded in evidence and analysis. Teams
that have adopted this approach recognized that the management of hypotheses
compels the team to base decisions on information rather than guesses.
HypoStage facilitates this by tracking the results of experiments, spikes, and
analytics, ensuring that when a decision is finally made, it is backed by data.

\section{Future Steps: Validation and Research Evolution}

While the development of HypoStage represents a significant step towards
resolving the challenges of decentralized architectural decision-making, it
marks the beginning of a new phase of research and validation. The path forward
involves rigorous empirical testing of the tool, the refinement of the
underlying methodologies, and the continued expansion of the pattern language
that supports ArchHypo.

\subsubsection{Empirical Validation in Diverse Contexts}

To fully understand the efficacy and generalizability of HypoStage, it is
essential to conduct extensive empirical studies. Future work must evaluate the
adoption of ArchHypo in companies and development teams with different
backgrounds and characteristics. This includes varying team sizes, domains, and
maturity levels. Furthermore, researchers must apply these patterns in other
projects to study their broader impact, address challenges, and refine
solutions to improve their adoption and effectiveness. These studies should aim
to quantify the reduction in decision-making bottlenecks and the improvement in
architectural quality.

\subsubsection{User Acceptance Evaluation}

The success of any developer tool depends on its acceptance by the practitioner
community. To systematically evaluate how engineering teams perceive and
interact with HypoStage, future interview and survey studies should be designed
to identify the factors that influence the adoption of the tool—such as
performance expectancy, effort expectancy, and social influence. Gathering this
data will help refine both the user experience and the feature set.

\subsubsection{Tool Improvement and Learning Curve Reduction}

A primary motivation for HypoStage was to mitigate the steep learning curve
associated with ArchHypo. Consequently, continuous investigation into tools and
methodologies to improve the implementation of ArchHypo represents an important
direction for research. Future iterations of the tool should focus on
intelligent features to further reduce this curve. This could include AI-driven
recommendation systems that suggest potential hypotheses based on project
characteristics or automated generation of technical plans based on historical
data.

\subsubsection{Evolution of the Framework via New Patterns}

The theoretical foundation of ArchHypo must also continue to evolve. Future
research should focus on identifying and documenting patterns for various types
of hypotheses. As the software landscape changes, new categories of uncertainty
emerge. There is a specific need for the exploration of common uncertainties
related to specific quality attributes, such as sustainability and usability.
Documenting these patterns will enrich the knowledge base available to users of
HypoStage, allowing them to leverage collective industry knowledge when
formulating their own hypotheses.

\subsubsection{Research on Process Guidelines}

Finally, the integration of ArchHypo into the broader software development
lifecycle remains a fertile area for research. The empirical study noted that
the introduction of specific process guidelines was a highly effective strategy
for managing uncertainty. Therefore, future studies can also investigate how
these guidelines can be adopted as an approach to dealing with uncertainty.
Understanding how to best weave hypothesis engineering into Agile, DevOps, and
other process methodologies will be crucial for the seamless adoption of
decentralized architectural decision-making.
