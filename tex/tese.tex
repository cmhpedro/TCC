\documentclass[a4paper,12pt,oneside,brazilian,english]{book}

\usepackage{imegoodies}
\usepackage[thesis]{imelooks}

% Diretórios onde estão as figuras; com isso, não é necessário (mas
% é permitido) colocar o caminho completo em \includegraphics. Note
% que a extensão nunca é necessária (mas é permitida), ou seja, o
% resultado é o mesmo com "\includegraphics{figuras/foto.jpeg}",
% "\includegraphics{foto.jpeg}", "\includegraphics{figuras/foto}"
% ou "\includegraphics{foto}".
\graphicspath{{figuras/},{fig/},{logos/},{img/},{images/},{imagens/}}

% Comandos rápidos para mudar de língua:
% \en -> muda para o inglês
% \br -> muda para o português
% \texten{blah} -> o texto "blah" é em inglês
% \textbr{blah} -> o texto "blah" é em português
\babeltags{br = brazilian, en = english}


%%%%%%%%%%%%%%%%%%%%%%%%%%%%%%%%%%%%%%%%%%%%%%%%%%%%%%%%%%%%%%%%%%%%%%%%%%%%%%%%
%%%%%%%%%%%%%%%%%%%%%%%%%%%%%%%%%% METADADOS %%%%%%%%%%%%%%%%%%%%%%%%%%%%%%%%%%%
%%%%%%%%%%%%%%%%%%%%%%%%%%%%%%%%%%%%%%%%%%%%%%%%%%%%%%%%%%%%%%%%%%%%%%%%%%%%%%%%

% O arquivo com os dados bibliográficos para biblatex; você pode usar
% este comando mais de uma vez para acrescentar múltiplos arquivos
\addbibresource{bibliografia.bib}

% Este comando permite acrescentar itens à lista de referências sem incluir
% uma referência de fato no texto (pode ser usado em qualquer lugar do texto)
%\nocite{bronevetsky02,schmidt03:MSc, FSF:GNU-GPL, CORBA:spec, MenaChalco08}
% Com este comando, todos os itens do arquivo .bib são incluídos na lista
% de referências
%\nocite{*}

\title{ArchHypo Tooling}[Enabling Practical Hypothesis Engineering for Software Architecture]
\translatedtitle{Title of the document}[a subtitle]

\author{Pedro Henrique Mariano Corrêa}

\orientador{José Gonçalves Lima Neto}
\coorientador{Prof. Dr. Paulo Roberto Miranda Meirelles}

\tipotese{
  tcc,
  programa={Ciência da Computação},
}

% A licença do seu trabalho. Use CC-BY, CC-BY-NC, CC-BY-ND, CC-BY-SA,
% CC-BY-NC-SA ou CC-BY-NC-ND para escolher a licença Creative Commons
% correspondente (o sistema insere automaticamente o texto da licença).
% Se quiser estabelecer regras diferentes para o uso de seu trabalho,
% converse com seu orientador e coloque o texto da licença aqui, mas
% observe que apenas TCCs sob alguma licença Creative Commons serão
% acrescentados ao BDTA. Se você tem alguma intenção de publicar o
% trabalho comercialmente no futuro, sugerimos a licença CC-BY-NC-ND.
%
%\direitos{CC-BY-NC-ND}
%
%\direitos{Autorizo a reprodução e divulgação total ou parcial deste
%          trabalho, por qualquer meio convencional ou eletrônico,
%          para fins de estudo e pesquisa, desde que citada a fonte.}
%
%\direitos{I authorize the complete or partial reproduction and disclosure
%          of this work by any conventional or electronic means for study
%          and research purposes, provided that the source is acknowledged.}
%
\direitos{CC-BY}

% Para gerar a ficha catalográfica, acesse https://fc.ime.usp.br/,
% preencha o formulário e escolha a opção "Gerar Código LaTeX".
% Basta copiar e colar o resultado aqui.
\fichacatalografica{}


%%%%%%%%%%%%%%%%%%%%%%%%%%%%%%%%%%%%%%%%%%%%%%%%%%%%%%%%%%%%%%%%%%%%%%%%%%%%%%%%
%%%%%%%%%%%%%%%%%%%%%%% AQUI COMEÇA O CONTEÚDO DE FATO %%%%%%%%%%%%%%%%%%%%%%%%%
%%%%%%%%%%%%%%%%%%%%%%%%%%%%%%%%%%%%%%%%%%%%%%%%%%%%%%%%%%%%%%%%%%%%%%%%%%%%%%%%

\begin{document}

%%%%%%%%%%%%%%%%%%%%%%%%%%% CAPA E PÁGINAS INICIAIS %%%%%%%%%%%%%%%%%%%%%%%%%%%%

% Aqui começa o conteúdo inicial que aparece antes do capítulo 1, ou seja,
% página de rosto, resumo, sumário etc. O comando frontmatter faz números
% de página aparecem em algarismos romanos ao invés de arábicos e
% desabilita a contagem de capítulos.
\frontmatter

\pagestyle{plain}

\onehalfspacing % Espaçamento 1,5 na capa e páginas iniciais

\maketitle % capa e folha de rosto

%%%%%%%%%%%%%%%% DEDICATÓRIA, AGRADECIMENTOS, RESUMO/ABSTRACT %%%%%%%%%%%%%%%%%%

% \begin{dedicatoria}
% 	Esta seção é opcional e fica numa página separada; ela pode ser usada para
% 	uma dedicatória ou epígrafe.
% \end{dedicatoria}

% Reinicia o contador de páginas (a próxima página recebe o número "i") para
% que a página da dedicatória não seja contada.
\pagenumbering{roman}

% Agradecimentos:
% Se o candidato não quer fazer agradecimentos, deve simplesmente eliminar
% esta página. A epígrafe, obviamente, é opcional; é possível colocar
% epígrafes em todos os capítulos. O comando "\chapter*" faz esta seção
% não ser incluída no sumário.
\chapter*{Acknowledgements}
% \epigrafe{Do. Or do not. There is no try.}{Mestre Yoda}

I would like to express my deep appreciation to Professor Paulo Meirelles and
Jose Neto for their invaluable support and advising throughout the development
of this capstone project. Their guidance was essential to realizing this work.

I am also grateful to the Eduardo Guerra group, whose research provided the
main foundation for this project. Thank you for the generosity of your time and
the insightful conversations that helped shape our approach.

Finally, I thank my family for their encouragement. A special thank you goes to
my mother for the enduring life lessons that have brought me to this moment. To
the friends I have made along this university journey, thank you for being a
part of my story.

\keywords{Software Architecture,Hypothesis Engineering,ArchHypo,Backstage}

\abstract{
	Modern software engineering faces a "Paradox of Software Architecture Decision-Making," where the industry strives for decentralized team autonomy but relies on architectural methods dependent on the tacit knowledge of experienced practitioners. While the ArchHypo framework addresses this by treating architectural decisions as testable hypotheses to manage uncertainty, its manual application creates a high cognitive load that hinders adoption. This work presents HypoStage, an open-source plugin for the Backstage Internal Developer Portal designed to operationalize the ArchHypo framework. By providing digital structures for uncertainty assessment, impact analysis, and technical planning, the tool creates the necessary scaffolding to lessen reliance on senior architects. The application of HypoStage to a start-up case study demonstrates its ability to facilitate decentralized, evidence-based architectural decision-making.
}


%%%%%%%%%%%%%%%%%%%%%%%%%%% LISTAS DE FIGURAS ETC. %%%%%%%%%%%%%%%%%%%%%%%%%%%%%

% Como as listas que se seguem podem não incluir uma quebra de página
% obrigatória, inserimos uma quebra manualmente aqui.
\cleardoublepage

% Todas as listas são opcionais; Usando "\chapter*" elas não são incluídas
% no sumário. As listas geradas automaticamente também não são incluídas por
% conta das opções "notlot" e "notlof" que usamos para a package tocbibind.

% Normalmente, "\chapter*" faz o novo capítulo iniciar em uma nova página, e as
% listas geradas automaticamente também por padrão ficam em páginas separadas.
% Como cada uma destas listas é muito curta, não faz muito sentido fazer isso
% aqui, então usamos este comando para desabilitar essas quebras de página.
% Se você preferir, comente as linhas com esse comando e des-comente as linhas
% sem ele para criar as listas em páginas separadas. Observe que você também
% pode inserir quebras de página manualmente (com \clearpage, veja o exemplo
% mais abaixo).
\newcommand\disablenewpage[1]{{\let\clearpage\par\let\cleardoublepage\par #1}}

% Nestas listas, é melhor usar "raggedbottom" (veja basics.tex). Colocamos
% a opção correspondente e as listas dentro de um grupo para ativar
% raggedbottom apenas temporariamente.
\bgroup
\raggedbottom

%%%%% Listas criadas manualmente

\chapter*{List of Abbreviations}

\begin{tabular}{rl}
	SA       & Software Architecture       \\
	ArchHypo & Architecture Hypothesis     \\
	IDP      & Internal Developer Portal   \\
	NFR      & Non-Functional Requirements \\
\end{tabular}

% Sumário (obrigatório)
\tableofcontents

\egroup % Final de "raggedbottom"

% Referências indiretas ("x", veja "y") para o índice remissivo (opcionais,
% pois o índice é opcional). É comum colocar esses itens no final do documento,
% junto com o comando \printindex, mas em alguns casos isso torna necessário
% executar texindy (ou makeindex) mais de uma vez, então colocar aqui é melhor.
\index{Inglês|see{Língua estrangeira}}
\index{Figuras|see{Floats}}
\index{Tabelas|see{Floats}}
\index{Código-fonte|see{Floats}}
\index{Subcaptions|see{Subfiguras}}
\index{Sublegendas|see{Subfiguras}}
\index{Equações|see{Modo matemático}}
\index{Fórmulas|see{Modo matemático}}
\index{Rodapé, notas|see{Notas de rodapé}}
\index{Captions|see{Legendas}}
\index{Versão original|see{Tese/Dissertação, versões}}
\index{Versão corrigida|see{Tese/Dissertação, versões}}
\index{Palavras estrangeiras|see{Língua estrangeira}}
\index{Floats!Algoritmo|see{Floats, ordem}}

%%%%%%%%%%%%%%%%%%%%%%%%%%%%%%%% CAPÍTULOS %%%%%%%%%%%%%%%%%%%%%%%%%%%%%%%%%%%%%

% Aqui vai o conteúdo principal do trabalho, ou seja, os capítulos que compõem
% a dissertação/tese. O comando mainmatter reinicia a contagem de páginas,
% modifica a numeração para números arábicos e ativa a contagem de capítulos.
\mainmatter

\pagestyle{mainmatter}

% Espaçamento simples
\singlespacing

% A introdução não tem número de capítulo, então os cabeçalhos também não
\pagestyle{unnumberedchapter}
\chapter{Introduction}

\section{Software Architecture and the Challenge of Rapid Evolution}

In the modern landscape of software engineering, Software Architecture (SA) has
evolved from a static blueprint into a continuous, dynamic process that is
critical for business survival. As systems grow in complexity, SA serves as the
structural foundation that ensures long-term sustainability, acting as a
strategic asset that dictates an organization's agility in fluctuating markets.

Traditionally, architecture was defined by \cite{Perry1992} tripartite model
comprising elements (processing, data, and connections), form (relationships
and weights), and rationale (justification). However, the modern pressure to
deliver quickly often obfuscates the rationale, causing the form to degrade
into what is known as a "Big Ball of Mud".

The core tension today lies in balancing velocity with stability. While
architecture was once front-loaded, maintaining integrity while enabling rapid
feature delivery is now crucial, particularly in fast-paced environments like
startups where requirements emerge frequently. Practitioners must balance speed
with quality to ensure adaptable architectures; failing to do so results in
technical debt that paralyzes future development.

\section{The Central Paradox: Decentralization vs. Tacit Knowledge}

A profound dissonance exists between the industry's organizational aspirations
and the reality of architectural design methods. Modern trends, such as DevOps
and Team Topologies, advocate for democratizing decision-making to empower
autonomous teams and accelerate value flow.

However, this creates a "Paradox of Software Architecture Decision-Making".
While the industry pushes for decentralized, team-driven decisions, academic
evidence shows that existing methods for deriving architecture heavily rely on
the tacit knowledge of experienced practitioners. This reliance assumes the
presence of seasoned architects who can intuitively synthesize requirements—a
resource that is often scarce or expensive.

This dependency creates bottlenecks and limits the applicability of these
methods in teams lacking senior members, effectively concentrating power and
contradicting the ethos of distributed ownership. Consequently, a
self-reinforcing cycle emerges: organizations attempt to decentralize to handle
fast-flowing requirements, but are obstructed by frameworks that inherently
require centralization, blocking the adoption of a true socio-technical
approach.

\section{The Philosophical Solution: ArchHypo}

To resolve this paradox, this work builds upon the principles of ArchHypo, a
technique that shifts the view of architecture from a set of facts to a set of
experiments. ArchHypo employs hypotheses engineering to manage the
uncertainties inherent in software architecture.

By explicitly documenting uncertainties, ArchHypo allows for the strategic
postponement of decisions. This aligns with the "Continuous Architecture"
principle of delaying design decisions until the "Responsible Moment"—the point
where the cost of deciding is outweighed by the cost of waiting, and sufficient
information is available to minimize risk. Rather than making premature
commitments that lead to brittle systems, teams formulate technical plans based
on hypothesis assessment to mitigate impact and reduce uncertainty. This
evidence-based process allows decision-making to be safely distributed among
team members.

\section{The Practical Contribution: From Theory to Tooling}

While ArchHypo offers a robust theoretical framework, its practical application
faces significant friction due to the high cognitive load required to shift
from a feature-centric to a hypothesis-centric mindset. Observational studies
indicate that without guidance, teams find the technique hard to learn,
particularly regarding mapping risks and defining action plans.

There is a strong need for tool support to manage these hypotheses and execute
action plans to lessen dependency on senior architects. To address this, the
primary practical contribution of this capstone is the design and development
of HypoStage. As an ArchHypo plugin for Backstage, HypoStage integrates
directly into the developer's existing workflow within an Internal Developer
Portal (IDP).

HypoStage operationalizes ArchHypo by providing digital structures for
uncertainty assessment, quality attribute tracking, and technical planning. By
embedding these capabilities into a tool, this work aims to bridge the gap
between theoretical decentralization and the practical reality of fast-paced
development, effectively resolving the paradox of architectural
decision-making.

\pagestyle{mainmatter}
\chapter{Theoretical and Architectural Basis}

This chapter establishes the theoretical foundations necessary for the
development of the proposed solution. It begins by contextualizing Software
Architecture within dynamic and agile environments, identifying the friction
between rapid delivery and architectural integrity. Subsequently, it delineates
the central problem motivating this work: the Paradox of Software Architecture
Decision-Making. To address this paradox, the chapter introduces the concepts
of Hypothesis Engineering and the ArchHypo framework, detailing its processes,
assessment metrics, and the pattern language that supports technical planning.
Finally, the chapter presents the practical gap identified in empirical
studies, justifying the necessity for the development of specific tooling to
support these practices.

\section{The Role of Software Architecture in Agile and High-Flow Environments}

Software Architecture (SA) is a fundamental concept in software development,
forming the structural foundation of systems and ensuring their long-term
sustainability. While early definitions, such as those by Perry and Wolf in
1992, characterized SA as a composition of elements, form, and rationale,
modern definitions have evolved to incorporate deeper insights into the
dynamics of system evolution. Contemporary definitions, such as those by
Richards and Ford, characterize SA as a high-level system structure
encompassing essential characteristics, design principles, and the decisions to
guide system adaptability and maintainability.

This expanded perspective highlights that SA is not merely a static blueprint
but a dynamic framework that guides decisions throughout the system life cycle
and aligns technical and business needs. However, the practical application of
architectural design is fraught with complexity, particularly regarding the
timing and certainty of decisions.

\subsection{The Challenge of Architecture in Agile Contexts}

Agile methodologies have successfully introduced practices like Test-Driven
Development (TDD) and refactoring, yet they often lack well-recognized and
agreed-upon approaches for architectural design. In many agile contexts, the
most recognized approach is to simply let the architecture "emerge" and refine
it over the life of the project. While this approach aims to reduce upfront
design, it often leads to a lack of established practice, meaning architectural
changes remain challenging within agile contexts, often necessitating
significant upfront design that contradicts core agile principles.

In rapidly evolving business environments, such as software startups, new
requirements continually reshape architectural demands. In these dynamic
scenarios, architecture anti-patterns—such as the "Big Ball of Mud"—may emerge,
posing risks to effective architectural evolution and resulting in systems that
struggle to maintain proper modularity and scalability. This anti-pattern often
emerges when immediate demands and incremental fixes override deliberate
architectural planning. Consequently, engineering teams must balance speed with
quality to ensure sustainable, adaptable architectures that meet evolving
business demands.

\subsection{Trends in Evolutionary Architecture}

To address the friction between agility and structure, concepts such as
\textit{Agile Architecture} and \textit{Continuous Architecture} have emerged.
These frameworks emphasize principles such as delaying design decisions and
architecting for change. The core philosophy is that decisions can be postponed
until they are absolutely necessary, ensuring that architectures are based on
facts rather than guesses.

However, determining which decisions to delay or where flexibility is
beneficial incurs significant challenges. Time emerges as a critical dimension
in architectural design, with iterative planning being essential, particularly
for solutions undergoing continuous evolution.

\subsection{The Necessity of Fast Flow}

Recent industry trends emphasize the concept of "fast flow," which refers to an
organization's ability to continuously and rapidly deliver software changes
that align with evolving business needs while preserving the system health and
architectural integrity. Achieving fast flow relies on autonomous, empowered
teams supported by self-service platforms that minimize operational blockers.

This concept is intrinsically linked to the adoption of socio-technical
architecture, which reflects the growing realization that architectural
decisions should not be confined to a select few architects but distributed
across development teams. The socio-technical approach promotes team autonomy
and empowerment, advocating for decision-making processes that include all team
members, regardless of their experience levels.

\section{The Paradox of Software Architecture Decision-Making}

Despite the industry's aspiration for decentralized, autonomous teams capable
of maintaining a "fast flow" of requirements, a significant theoretical and
practical barrier exists. This research identifies this barrier as the
\textbf{Paradox of Software Architecture Decision-Making}.

\subsection{The Central Paradox}

The Paradox of Software Architecture Decision-Making highlights the conflicting
dynamics between industry reports advocating for decentralized, team-driven
architectural decisions to enable a fast flow of requirements, and academic
research indicating that existing methods for deriving SA still heavily depend
on the tacit knowledge of experienced practitioners.

While the industry moves toward distributed architectural ownership, the
current methods and frameworks continue to reinforce centralized
decision-making. This paradox creates a self-reinforcing cycle: while
organizations may aim for a decentralized approach to support a fast flow of
requirements, they remain dependent on expert-driven methods that, by design,
limit the involvement of less experienced team members.

\subsection{Reliance on Tacit Knowledge}

A systematic mapping study by Souza et al. (2019) examined methods and
practices for deriving architectural models from requirements specifications. A
key finding was that existing methods strongly relied on experienced
practitioners' tacit knowledge to derive the architectural definitions for a
software system. This reliance on the intuition and expertise of architects
makes it difficult for teams without such expertise to adopt these methods
effectively.

This dependency creates bottlenecks that limit an effective decentralization of
decision-making. As a result, the shift towards a more distributed
decision-making model is obstructed by frameworks that inherently require
centralization, blocking the practical adoption of a socio-technical
architecture approach.

\subsection{Deficiencies in Existing Methods}

Beyond the reliance on tacit knowledge, the systematic mapping study
highlighted several other gaps in existing architectural derivation methods:

\begin{itemize}
	\item \textbf{Lack of Tool Support:} About a third of analyzed architectural derivation approaches lack tool support. The study noted that only 30.7\% of methods provided decision-making support.
	\item \textbf{Inadequate Support for Non-Functional Requirements (NFRs):} Although NFRs such as performance, scalability, and security are essential, many approaches focus primarily on functional requirements, neglecting NFRs in architectural decision-making processes.
	\item \textbf{Limited Empirical Validation:} Many methods have not been sufficiently validated in real-world, dynamic environments. Over half lack explicit evaluation methods.
\end{itemize}

This points to a clear need for tools to aid architects, suggesting patterns
and supporting decision-making to reduce dependency on seasoned experts.

\section{Hypothesis Engineering and Uncertainty Management}

To resolve the paradox of decision-making and enable decentralized
architecture, it is necessary to move away from reliance on tacit intuition
toward a more explicit management of the unknowns. This is achieved through
\textbf{Hypothesis Engineering}.

\subsection{The Concept of Hypothesis Engineering}

Hypothesis Engineering offers a philosophical approach to managing uncertainty
by treating assumptions as testable hypotheses. It is defined as a process of
continuously validating product assumptions, transforming them into hypotheses,
prioritizing and testing them following the scientific method to support or
refute them.

In this context, the word "hypothesis" focuses on business assumptions that
should be evaluated before start-ups develop their business models. However,
this concept is extended here to manage the uncertainty of architectural
decisions.

\subsection{Distinguishing Uncertainty from Risk}

A critical theoretical distinction must be made between \textbf{risk} and
\textbf{uncertainty}. Risk involves evaluating the probability and impact of
specific events; it represents the likelihood of an event occurring and its
potential consequences.

In contrast, \textbf{uncertainty} refers to the lack of information necessary
for architectural decisions. Uncertainty represents the lack of information to
make a decision; in cases where partial information is available, the
uncertainty is related to what is still unknown. ArchHypo is based on the
premise that uncertainties related to the software architecture are natural in
all stages of a software project and that, instead of resisting them, a better
approach would be to embrace and manage them.

\subsection{Characteristics of an Architectural Hypothesis}

An architectural hypothesis represents any uncertain statement relevant to the
software architecture design. Its primary characteristic is that it must be
\textbf{falsifiable}—that is, the possibility of proving it false exists.

Unlike a requirement statement, which is usually assumed as true when written,
using the word "hypothesis" makes it clear to the whole team that it represents
uncertainty. Hypotheses must be shared with the development team so that
everyone is aware of the uncertainties that can affect architectural decisions.
This explicit documentation allows the team to systematically test and validate
assumptions, gather data, and adjust their approach as needed.

\section{The ArchHypo Framework}

ArchHypo is a technique that uses hypothesis engineering to manage
uncertainties related to software architecture and enhance decision-making
processes. It provides a structured framework for software architects and
developers, enabling them to make uncertainties explicit and manageable.

\subsection{Sources of Uncertainty}

When applying ArchHypo, a team focuses specifically on hypotheses that can
affect the software architecture. The most common sources of uncertainty are in
the \textbf{requirements} and the \textbf{solutions}.

\begin{itemize}
	\item \textbf{Requirements Uncertainty:} This occurs when a requirement is based on limited evidence or lacks important details. For instance, uncertainty regarding the number of simultaneous requests a system must handle.
	\item \textbf{Solution Uncertainty:} This relates to the unknown consequences of current architecture components or candidate solutions. For example, uncertainty regarding whether a specific library is compatible with a required protocol.
\end{itemize}

\subsection{Assessment: Uncertainty Level and Impact}

Once a hypothesis is identified, it is assessed based on two independent
dimensions:

\begin{itemize}
	\item \textbf{Uncertainty Level:} This reflects how far the team is from proving that a hypothesis is true or false. The uncertainty would be high if there is a lack of information to estimate probability or when alternatives have a similar chance to happen.
	\item \textbf{Impact:} This measures the effort required to transition to a different alternative. It represents the consequences and costs that the uncertainty can cause.
\end{itemize}

This assessment is typically performed qualitatively by the team, using a scale
(e.g., a five-point Likert scale: Very Low to Very High). These assessments aim
to provoke reflection about the hypotheses, allowing a qualitative comparison
among them and helping in choosing techniques for handling the respective
uncertainty.

\subsection{The Technical Plan}

The core operational mechanism of ArchHypo is the formulation of a
\textbf{Technical Plan} based on each hypothesis' assessment. This plan is
derived from the pattern \textit{Plan for Responsible Moments}.

The plan can include actions aiming to:

\begin{itemize}
	\item Definitely accept or refute the hypothesis.
	\item \textbf{Reduce the uncertainty:} Giving the team more confidence to move forward with a decision.
	\item \textbf{Reduce the impact:} Allowing the team to "stay longer with the uncertainty" by isolating the affected areas or increasing flexibility.
	\item Define criteria (triggers) to postpone handling the hypothesis until a specific
	      condition is met.
\end{itemize}

\subsection{Supporting Practices and Patterns}

The technical plan utilizes a specific pattern language to address
uncertainties. Key patterns include:

\begin{itemize}
	\item \textbf{Architectural Spike:} Small technical experiments in which working software is created to prove or disprove the feasibility of a specific hypothesis.
	\item \textbf{Tracer Bullet:} An architecturally significant functionality that helps to exercise and demonstrate an end-to-end path inside the architecture, aiming to evaluate how new technologies could be integrated. This pattern provides a concrete way to design the basic application architecture.
	\item \textbf{Software Analytics:} Identifying relevant metrics to collect and analyze to assess and monitor a given quality attribute of the system.
	\item \textbf{Architectural Trigger:} Defines conditions that trigger architectural investigations which may lead to adding tasks to the backlog. For example, having a given number of simultaneous accesses might fire a trigger related to scalability.
	\item \textbf{Development Guidelines:} Small adjustments in the development process to deal with recurrent uncertainties. These include:
	\item \textbf{Protective Guideline:} Defining programming practices to be followed or avoided to not limit options for an architectural decision being postponed.
	\item \textbf{Bring the Specialist:} Involving individuals with the right skills or knowledge in activities where this expertise can reduce the uncertainty.
	\item \textbf{Plan for Preparation:} Introducing steps to obtain information before activities that recurrently have an associated uncertainty.
	\item \textbf{Quality Checkpoint:} Introducing a verification activity after the development of an artifact to verify if the desired quality is present.
\end{itemize}

\section{The Practical Gap and the Need for Tooling}

While the theoretical framework of ArchHypo is robust, empirical evidence
suggests that its manual application presents significant challenges that
hinder widespread adoption.

\subsection{Adoption Barriers}

An empirical study on ArchHypo reported that while the technique provided a
structured approach to dividing architectural work, the team identified the
learning curve and process adjustments required for adoption as significant
challenges. Specifically, participants found the technique hard to learn,
particularly in mapping risks, specifying hypotheses, and defining action
plans.

Team members mentioned difficulty in "mapping the scenarios with risk" to the
hypothesis and "definition of the actions" to handle the hypothesis. One
participant explicitly noted that "train[ing] the team a little more on how to
find, map risk scenarios, and define the necessary actions... would make the
team less dependent on the team of architects".

\subsection{The Necessity of Tool Support}

The identified difficulties highlight a strong need for better guidance and,
crucially, \textbf{tool support to manage hypotheses and execute action plans},
which could lessen team dependency on architects. The current literature notes
that about a third of analyzed architectural derivation approaches lack tool
support.

Tools that provide more guidance on using the technique can be developed to
reduce its learning curve. Without effective tools to aid architects,
suggesting patterns and supporting decision-making, the dependency on seasoned
experts remains a bottleneck.

\subsection{Integration with Developer Platforms}

To bridge this gap, researchers have suggested that plugins of existing tools
could introduce support for hypotheses management. By integrating with project
management platforms or developer portals, a tool can facilitate seamless
adoption into diverse development workflows.

The development of \textbf{HypoStage}, a dedicated software tool, aims to
operationalize the ArchHypo framework. It intends to allow teams to explicitly
document and manage architectural hypotheses, assess them based on impact and
uncertainty, and define and track technical action plans. This technological
intervention addresses the core of the "Paradox of Software Architecture
Decision-Making" by providing the structural scaffolding necessary for less
experienced practitioners to engage in decentralized, high-quality
architectural decision-making.

\chapter{The HypoStage Tooling Solution}

This chapter presents the practical contribution of this research: the
conception, architectural design, and implementation of \textbf{HypoStage}.
Based on the theoretical foundations established in the previous chapter and
the empirical gaps identified in the literature regarding the ArchHypo
framework, this chapter details the development of a software tool designed to
operationalize decentralized architectural decision-making. The chapter begins
by justifying the necessity of a technological intervention to resolve the
Paradox of Software Architecture Decision-Making. It then proceeds to describe
the solution's integration within the Backstage internal developer portal,
details the functional requirements for managing the hypothesis lifecycle, and
concludes with a technical examination of the software architecture and the
open-source strategy adopted for the project.

\section{Justification for Tooling Support}

The theoretical framework of ArchHypo, as discussed in Chapter 2, provides a
robust methodology for managing architectural uncertainty through Hypothesis
Engineering. However, the transition from theory to practice presents
significant hurdles. Empirical research conducted by \cite{ArchHypo}. on the
application of ArchHypo has demonstrated clear benefits, such as improved
collaboration and a structured approach to architectural work, yet it
simultaneously revealed significant adoption challenges.

\subsection{The Dilemma of Manual Application}

The primary barrier to the widespread adoption of Hypothesis Engineering in
software architecture lies in the cognitive load and process overhead imposed
on development teams. Participants in empirical studies found the technique
hard to learn, particularly in mapping risks, specifying hypotheses, and
defining action plans. Specifically, the abstract nature of translating vague
architectural uncertainties into testable hypotheses proved to be a cognitive
bottleneck. The study highlighted that teams struggled with mapping the
scenarios with risk and properly articulating the necessary technical
interventions.

Furthermore, without structured guidance, the execution of the framework often
defaulted back to the most experienced members of the team, reinforcing the
very centralization the method aims to dismantle. The feedback from
practitioners was explicit regarding the solution to this problem: there is a
strong need for tool support to manage hypotheses and execute action plans,
which could lessen team dependency on architects. Without effective tools to
aid architects—suggesting patterns and supporting decision-making—the
dependency on seasoned experts remains a bottleneck, perpetuating the
centralization of architectural authority.

\subsection{Technological Intervention as a Scaffolding Mechanism}

This capstone project posits that the Paradox of Software Architecture
Decision-Making — where the desire for decentralized autonomy conflicts with
the reliance on centralized tacit knowledge—cannot be resolved through process
changes alone. It requires a technological intervention. The development of a
dedicated software tool aims to operationalize the ArchHypo framework.

By embedding the methodological rules, assessment scales, and technical
patterns directly into a software interface, the tool provides the necessary
structural scaffolding for less experienced practitioners. It intends to allow
teams to explicitly document and manage architectural hypotheses, assess them
based on impact and uncertainty, and define and track technical action plans.
Consequently, the tool acts as a mechanism for knowledge distribution, enabling
the decentralization of high-quality architectural decision-making.

\section{HypoStage and the Backstage Platform Strategy}

To address the identified needs, this project introduces \textbf{HypoStage}, a
software solution designed to integrate architectural hypothesis management
into the daily workflow of engineering teams.

\subsection{The HypoStage Solution}

HypoStage is defined as an ArchHypo plugin for Backstage. It is not merely a
documentation repository but an active management tool that integrates
architectural hypothesis management into your Backstage environment, enabling
teams to document, track, and validate architectural decisions effectively. The
tool is designed to guide users through the lifecycle of a hypothesis, from
elicitation to validation or refutation, ensuring that uncertainty is treated
as a first-class citizen in the software delivery process.

A key design principle of HypoStage is its capability to seamlessly integrates
with Backstage's catalog and entity system. This ensures that architectural
hypotheses are not abstract entities but are directly linked to the software
components, APIs, and resources they affect, promoting traceability and
context-aware decision-making.

\subsection{Rationale for the Backstage Ecosystem}

The decision to implement HypoStage as a plugin for \textbf{Backstage}—an
open-source IDP—is strategic. To bridge the gap identified in the literature,
researchers have suggested that plugins of existing tools could introduce
support for hypotheses management. By integrating with project management
platforms or developer portals, a tool can facilitate seamless adoption into
diverse development workflows.

Backstage was selected as the host platform because it serves as a centralized
hub for software development teams, aggregating infrastructure, services, and
documentation. By placing hypothesis management within the IDP, HypoStage
ensures that architectural decision-making occurs in the same environment where
development happens. This visibility is crucial for decentralization, as it
democratizes access to architectural rationale. Furthermore, the modular
architecture of HypoStage allows it to be integrated with existing developer
portals, facilitating seamless adoption, leveraging Backstage's extensible
plugin architecture to reach a wide audience of practitioners.

\section{Functional Requirements and Implementation}

HypoStage implements specific functional requirements derived from the ArchHypo
framework to support the complete lifecycle of hypothesis engineering. These
functionalities cover documentation, assessment, planning, and visualization.

\subsection{Explicit Hypothesis Management and Tracking}

The core functionality of HypoStage is Hypothesis Management, which allows
users to create, edit, and track architectural hypotheses with detailed
metadata. The system enforces a structured format for elicitation, requiring
clear statements that define the uncertainty. The implementation provides a
robust form interface that captures essential data points such as the
Hypothesis Statement, Source Type (e.g., Requirements, Solution), and
associated Entity References from the catalog.

Furthermore, the tool supports Status Tracking, allowing teams to monitor
hypothesis lifecycle from creation to validation. The system supports multiple
statuses for hypotheses, such as Open, In Review, Validated, Discarded, and
Trigger-Fired. This explicit tracking ensures that architectural uncertainties
are not forgotten but are actively managed until resolution.

\subsection{Uncertainty and Impact Assessment}

A critical component of the ArchHypo framework is the quantitative and
qualitative assessment of hypotheses. HypoStage implements Uncertainty
Assessment functionality to evaluate hypothesis uncertainty using Likert scale
ratings.

The interface requires users to provide an Uncertainty Rating and an Impact
Rating, both utilizing a 1-5 scale. These ratings map to values ranging from
Very Low (1) to Very High (5). This structured assessment forces teams to
critically evaluate how far they are from validating a hypothesis and what the
potential consequences of failure are, directly addressing the difficulty
participants faced in mapping risks in manual implementations.

\subsection{Technical Planning and Quality Correlation}

To move from assessment to action, HypoStage provides Technical Planning
capabilities. The tool allows teams to create and manage technical planning
items linked to hypotheses. Users can define specific actions—such as
Architectural Spike, Tracer Bullet, or Prototype — assigned to specific
entities with target dates and expected outcomes. This directly supports the
objective to define and track technical action plans associated with each
hypothesis.

Additionally, the tool implements Quality Attributes Tracking, enabling users
to associate hypotheses with specific quality attributes. The system supports a
comprehensive list of attributes, such as Performance, Security, Scalability,
and Maintainability. This feature ensures that architectural decisions are
explicitly linked to the non-functional requirements they impact, fostering a
quality-driven architectural culture.

\subsection{Visualization and Evidentiary Support}

To support decision-making over time, HypoStage includes Visualization features
to track hypothesis evolution and validation status through interactive charts.
The system renders a temporal view of how uncertainty and impact ratings change
as technical plans are executed, providing empirical evidence of risk
reduction.

Finally, to ground decisions in facts rather than intuition, the tool supports
the inclusion of Evidence URLs. The system allows teams to add supporting
documentation links, such as external test results, POC repositories, or vendor
documentation, thereby ensuring that the validation of a hypothesis is
auditable and transparent.

\section{Software Engineering Architecture}

The engineering of HypoStage follows modern software development practices,
utilizing a decoupled architecture that aligns with the Backstage plugin
ecosystem. The solution is divided into two main packages: the frontend plugin
and the backend plugin.

\subsection{Component Architecture}

The system architecture is composed of distinct frontend components and backend
services.

\subsubsection{Frontend Architecture}

The frontend is built using React and integrates with the Backstage core API.

\subsubsection{Backend Architecture}

The backend logic is encapsulated in the backend package.

\begin{itemize}
	\item HypothesisService: The core service for hypothesis management. As seen in this
	      service manages business logic and data persistence. It utilizes database
	      transactions to ensure data integrity when creating hypotheses and logging
	      associated lifecycle events simultaneously.
	\item Router: The router module uses express-promise-router to define RESTful
	      endpoints, handling request validation via Zod schemas before invoking the
	      service layer.
	\item Persistence: The system uses Knex.js for database interactions, including
	      database migrations for schema setup which define the structure for hypothesis,
	      technicalPlanning, and hypothesisEvents tables.
\end{itemize}

\subsection{Design for Integration and Modularity}

The project adheres to the goal of ensuring modularity and integrability. The
plugin is registered within the Backstage backend system using the
createBackendPlugin factory, allowing it to inject dependencies such as logger,
database, httpAuth, and catalogService.

This dependency injection model is crucial for the tool's operation as a
socio-technical enabler. For instance, the HypothesisService interacts with the
CatalogService to fetch entity references, ensuring that hypotheses are tightly
coupled to the actual software components registered in the organization's
ecosystem. This design allows HypoStage to be used as a standalone application
(in a dev environment) or integrated with existing developer portals,
fulfilling the requirement for seamless adoption.

\section{Open Source Commitment and Documentation}

Recognizing that the challenges of architectural decision-making are universal,
HypoStage is positioned as a contribution to the wider software engineering
community.

\subsection{Licensing and Distribution}

The project is distributed as open-source software under the LGPL-3.0 license
(GNU Lesser General Public License v3.0). This licensing model was chosen to
balance the freedom of use with the requirement that modifications to the
plugin library itself remain open source, thereby encouraging community
contribution back to the core tool while allowing integration into proprietary
Backstage instances.

\subsection{Documentation for Adoption}

To mitigate the learning curve not just of the method, but of the tool itself,
comprehensive documentation has been developed. The README.md serves as the
primary entry point, detailing Installation, Configuration, and Usage guides.
It explains how to configure the frontend routes and backend plugins, ensuring
that organizations can adopt the tool with minimal friction. By providing clear
API Reference and Features descriptions, the project aims to lower the barrier
to entry, facilitating the shift from centralized, tacit architectural
management to a transparent, decentralized, and hypothesis-driven approach.

\chapter{Applying HypoStage to the Catch Solve Start-up}

This chapter demonstrates the practical application of the HypoStage tool by
revisiting the "Catch Solve" case study presented by \cite{UseCase}. In the
original study, the ArchHypo technique was applied manually through meetings
and interviews to manage architectural uncertainty in a web testing start-up.
This chapter reconstructs that experience, illustrating how HypoStage would be
used to digitize, manage, and track the same architectural hypotheses,
assessments, and technical plans described in the literature.

By mapping the real-world decisions faced by Catch Solve to the specific
functional capabilities of HypoStage detailed in Chapter 3, we provide a
concrete walkthrough of the tool's workflow from elicitation to resolution.

\section{System Context and Entity Registration}

Before managing hypotheses, the software system under analysis must be
contextually defined. As described in Chapter 3, HypoStage integrates with the
Backstage Catalog to link hypotheses directly to software entities.

In the context of Catch Solve, the primary system is a platform that offers
testing and monitoring services for web applications. The first step in the
HypoStage workflow is navigating to the Catch Solve Platform entity within the
developer portal. By anchoring the hypothesis to this entity, all subsequent
architectural decisions remain traceable to the specific component they affect.

\section{Phase 1: Hypothesis Elicitation}

The first phase involves capturing the architectural uncertainties identified
by the development team. In the manual study, this was done via a meeting with
the technical lead. In HypoStage, this is achieved using the Hypothesis
Management interface to create structured records.

\begin{figure}
	\centering
	\includegraphics[width=0.9\textwidth]{contents/images/chapter-4-hypothesis-list}
	\caption{The list of hypotheses created for the Catch Solve case study in HypoStage}
	\label{fig:chapter-4-hypothesis-list}
\end{figure}

\subsection{Defining the Availability Hypothesis}

One of the initial concerns raised by Catch Solve was that "The lack of
redundancy can cause problems with application availability for the customers".

Using HypoStage, a user would create a new hypothesis with the following
structured data:

Statement: The lack of redundancy can cause problems with application
availability for the customers.

Quality Attribute: Availability.

Source Type: Solution (referring to the current architectural limitation).

Status: Open.

\subsection{Defining the Scalability Hypothesis}

A second major concern was the manual creation of tests, which limited the
start-up's ability to scale. The team hypothesized that "Test templates can be
used to generate tests for different applications".

In HypoStage, this entry would be recorded as:

Statement: Test templates can be used to generate tests for different
applications.

Quality Attribute: Reusability.

Source Type: Requirement (focusing on the need to optimize the test creation
process).

Status: Open.

\section{Phase 2: Uncertainty and Impact Assessment}

Once the hypotheses are documented, HypoStage requires a quantitative
assessment to prioritize them. The tool utilizes a 5-point Likert scale for
both Uncertainty and Impact.

\subsection{Assessing Availability}

In the case study, the team realized that implementing redundancy was not
immediately critical because the application was not mission-critical and
customers had not complained. Furthermore, the team already knew how to
implement redundancy if needed (low uncertainty).

HypoStage Input:

Uncertainty Rating: 1 (Very Low) — Rationale: The solution is known.

Impact Rating: 2 (Low) — Rationale: Current customer base tolerates occasional
downtime.

\subsection{Assessing Test Templates}

Conversely, the "Test Templates" hypothesis presented a significant challenge.
The founder did not know how to implement parameterized tests (high
uncertainty) and recognized that failure here would affect the core execution
component (high impact).

HypoStage Input:

Uncertainty Rating: 5 (Very High) — Rationale: Implementation path is unknown.

Impact Rating: 5 (Very High) — Rationale: Affects core business scalability.

The Visualization feature of HypoStage would immediately highlight this
contrast.

\section{Phase 3: Technical Planning}

HypoStage moves beyond simple documentation by allowing teams to attach
executable Technical Plans to each hypothesis.

\subsection{Planning for Availability (The Trigger)}

Since the availability hypothesis had low impact and uncertainty, the decision
was to postpone the architectural change. The plan was to monitor the system
and revisit the decision only if the customer base grew.

In HypoStage, a Technical Planning Item is added to this hypothesis:

Action Type: Architectural Trigger.

Description: Monitor unavailability reports. Revisit redundancy architecture if
new customer count exceeds threshold.

\subsection{Planning for Test Templates (The Spike)}

For the high-risk "Test Templates" hypothesis, the team needed to reduce
uncertainty through experimentation. The study describes using a "Tracer
Bullet" to create a reusable template instance and an "Architecture Spike" to
investigate existing test suites.

In HypoStage, two planning items are created:

Action Type: Tracer Bullet.

Description: Create a single reusable test template instance to valid
integration with the current runner.

Action Type: Architectural Spike.

Description: Analyze existing customer test suites to identify recurrent
verification patterns.

\begin{figure}
	\centering
	\includegraphics[width=0.9\textwidth]{contents/images/chapter-4-tech-plan-1}
	\caption{Creating a technical plan for the Test Templates hypothesis}
	\label{fig:chapter-4-tech-plan-1}
\end{figure}

\begin{figure}
	\centering
	\includegraphics[width=0.9\textwidth]{contents/images/chapter-4-tech-plan-2}
	\caption{Technical plans associated with the Test Templates hypothesis}
	\label{fig:chapter-4-tech-plan-2}
\end{figure}

\section{Phase 4: Execution, Evidence, and Evolution}

The power of HypoStage lies in its lifecycle tracking. As time progresses, the
team returns to the tool to update the status based on the results of their
technical plans.

\subsection{Ten Months Later: Re-evaluating Test Templates}

The case study reported that after 10 months, students had implemented proofs
of concept (PoC), and a feature to check for broken URLs was successfully
introduced.

Using HypoStage, the team updates the hypothesis to reflect this progress:

Evidence URLs: The user adds links to the student PoC repositories and the pull
request for the "Broken URL" feature.

Re-Assessment:

Uncertainty: Downgraded from 5 (Very High) to 2 (Low).

Impact: Downgraded from 5 (Very High) to 2 (Low).

Status: Changed from "Open" to "Validated".

\begin{figure}
	\centering
	\includegraphics[width=0.9\textwidth]{contents/images/chapter-4-update-1}
	\caption{Updating the hypothesis with evidence and re-assessment after 10 months}
	\label{fig:chapter-4-update-1}
\end{figure}

\subsection{Handling the Availability Trigger}

During the same period, the availability hypothesis remained stable. However,
the team implemented some database redundancy for maintainability reasons,
which had the side effect of improving availability.

In HypoStage, this evolution is recorded by adding a comment to the hypothesis
history or linking a new "Maintainability" hypothesis that references the
original "Availability" concern. This captures the "side effect" nature of the
architectural evolution described in the study.

\begin{figure}
	\centering
	\includegraphics[width=0.9\textwidth]{contents/images/chapter-4-update-2}
	\caption{The updated hypothesis status and assessment reflecting the progress made}
	\label{fig:chapter-4-update-2}
\end{figure}

\section{Conclusion}

This walkthrough illustrates how HypoStage transforms the abstract ArchHypo
framework into a concrete, manageable workflow. By using the Catch Solve data,
we demonstrated how the tool supports the full lifecycle of architectural
decision-making:

Elicitation: Converting vague concerns into structured Hypothesis Entities
linked to the Backstage Catalog.

Assessment: Using the Likert Scale Interface to visually distinguish between
"postponable" decisions (Availability) and "critical" investigations (Test
Templates).

Planning: Assigning specific Technical Plans (Triggers vs. Spikes) to
operationalize the response to uncertainty.

Tracking: Using Evidence URLs and Status Updates to provide an audit trail of
how uncertainty was reduced over time.

While the original study relied on periodic meetings and manual documentation,
HypoStage enables this process to occur asynchronously and continuously within
the developer's native environment, ensuring that architectural knowledge
remains visible, accessible, and actionable.

\chapter{Conclusion: Contributions and Future Directions}

\section{Summary of Contributions: Filling the Tooling Void}

This capstone project addresses a fundamental operational void in modern
software architecture: the critical absence of specialized tooling to manage
architectural uncertainty. While the theoretical foundations for decentralized
decision-making exist in frameworks like ArchHypo, their practical application
has been stifled by a reliance on manual, high-friction processes that
development teams struggle to sustain.

The primary contribution of this work is the delivery of the missing
technological link: **HypoStage**.

HypoStage is a purpose-built software artifact designed to operationalize
Hypothesis Engineering. It transforms the abstract principles of architectural
uncertainty—previously managed through ad-hoc documentation or the tacit
knowledge of senior architects—into a concrete, structured digital workflow. By
automating the lifecycle of hypothesis elicitation, assessment, and technical
planning, the tool removes the administrative burden that has historically
blocked the adoption of evidence-based architectural methods.

Recognizing that modern engineering organizations are consolidating their
operations into Internal Developer Portals (IDPs), HypoStage was not built in
isolation but as a fully integrated plugin for **Backstage**. This strategic
architectural choice ensures that hypothesis management is embedded directly
into the developer's "Golden Path." By placing the tool where developers
already manage their services and deployments, HypoStage bridges the gap
between architectural intent and daily engineering execution, providing the
necessary scaffolding for teams to practice continuous, decentralized
architecture effectively.

\section{Results: Insights Empowered by the Tool}

The development of HypoStage is grounded in the proven empirical benefits of
the ArchHypo technique. The tool is designed not merely to record data, but to
amplify and sustain the positive outcomes observed in industrial applications
of the methodology. By facilitating the management of architectural hypotheses,
the tool unlocks several critical advantages for software development projects.

\subsubsection{Support for Prioritization and Risk Management}

One of the most significant impacts of applying this methodology is the
enhancement of risk visibility. In complex projects, architectural risks are
often opaque or implicit. The structured elicitation and assessment of
hypotheses supported by the tool allow for a clearer visualization of the risks
that could threaten the final delivery of the project.

By making these risks explicit and quantifiable through impact and uncertainty
scores, teams can prioritize their efforts more effectively, focusing on the
most critical uncertainties that could jeopardize project success.

\subsubsection{Sustainable Task Distribution and Planning}

The methodology provides a robust framework for managing the temporal dimension
of architecture. Rather than treating architecture as a monolithic upfront
phase, the approach supported by HypoStage enables a structured approach to
dividing the architectural work through iterations. This allows for allocating
and scheduling architectural tasks throughout project iterations, ensuring that
architectural evolution is continuous and integrated with feature development.
This distribution prevents the accumulation of unmanaged technical debt and
ensures that architectural work is prioritized alongside functional
requirements.

\subsubsection{Sustainable Development and Quality}

The systematic management of uncertainty contributes directly to the
sustainability of the development process. Empirical evidence suggests that the
adoption of this approach benefits projects by providing predictability,
security, transparency, and a sustainable pace. Furthermore, the technique has
been shown to contribute significantly to decision-making efficiency. By
reducing the chaos associated with unforeseen architectural blockers, teams can
maintain a steady velocity and deliver higher-quality software that meets both
functional and non-functional requirements.

\subsubsection{Reduction of Upfront Design}

A core tenet of modern agile architecture is the avoidance of heavy upfront
design, which often leads to waste and rigidity. The ArchHypo technique,
operationalized by the tool, allows projects to avoid an upfront architectural
design. It achieves this by providing a safety net that supports the strategic
postponement of decisions while addressing their potential impact. By
identifying which decisions can be safely delayed and monitoring their
associated risks, teams can maintain agility and keep their options open until
the "Most Responsible Moment" arises.

\subsubsection{Informed Decision-Making}

Finally, the ultimate goal of the tool is to elevate the quality of
architectural decisions. By moving away from intuition-based choices, the
methodology ensures that decisions are grounded in evidence and analysis. Teams
that have adopted this approach recognized that the management of hypotheses
compels the team to base decisions on information rather than guesses.
HypoStage facilitates this by tracking the results of experiments, spikes, and
analytics, ensuring that when a decision is finally made, it is backed by data.

\section{Future Steps: Validation and Research Evolution}

While the development of HypoStage represents a significant step towards
resolving the challenges of decentralized architectural decision-making, it
marks the beginning of a new phase of research and validation. The path forward
involves rigorous empirical testing of the tool, the refinement of the
underlying methodologies, and the continued expansion of the pattern language
that supports ArchHypo.

\subsubsection{Empirical Validation in Diverse Contexts}

To fully understand the efficacy and generalizability of HypoStage, it is
essential to conduct extensive empirical studies. Future work must evaluate the
adoption of ArchHypo in companies and development teams with different
backgrounds and characteristics. This includes varying team sizes, domains, and
maturity levels. Furthermore, researchers must apply these patterns in other
projects to study their broader impact, address challenges, and refine
solutions to improve their adoption and effectiveness. These studies should aim
to quantify the reduction in decision-making bottlenecks and the improvement in
architectural quality.

\subsubsection{User Acceptance Evaluation}

The success of any developer tool depends on its acceptance by the practitioner
community. To systematically evaluate how engineering teams perceive and
interact with HypoStage, future interview and survey studies should be designed
to identify the factors that influence the adoption of the tool—such as
performance expectancy, effort expectancy, and social influence. Gathering this
data will help refine both the user experience and the feature set.

\subsubsection{Tool Improvement and Learning Curve Reduction}

A primary motivation for HypoStage was to mitigate the steep learning curve
associated with ArchHypo. Consequently, continuous investigation into tools and
methodologies to improve the implementation of ArchHypo represents an important
direction for research. Future iterations of the tool should focus on
intelligent features to further reduce this curve. This could include AI-driven
recommendation systems that suggest potential hypotheses based on project
characteristics or automated generation of technical plans based on historical
data.

\subsubsection{Evolution of the Framework via New Patterns}

The theoretical foundation of ArchHypo must also continue to evolve. Future
research should focus on identifying and documenting patterns for various types
of hypotheses. As the software landscape changes, new categories of uncertainty
emerge. There is a specific need for the exploration of common uncertainties
related to specific quality attributes, such as sustainability and usability.
Documenting these patterns will enrich the knowledge base available to users of
HypoStage, allowing them to leverage collective industry knowledge when
formulating their own hypotheses.

\subsubsection{Research on Process Guidelines}

Finally, the integration of ArchHypo into the broader software development
lifecycle remains a fertile area for research. The empirical study noted that
the introduction of specific process guidelines was a highly effective strategy
for managing uncertainty. Therefore, future studies can also investigate how
these guidelines can be adopted as an approach to dealing with uncertainty.
Understanding how to best weave hypothesis engineering into Agile, DevOps, and
other process methodologies will be crucial for the seamless adoption of
decentralized architectural decision-making.

\nocite{*}

%%%%%%%%%%%%%%% SEÇÕES FINAIS (BIBLIOGRAFIA E ÍNDICE REMISSIVO) %%%%%%%%%%%%%%%%

% O comando backmatter desabilita a numeração de capítulos.
\backmatter

\pagestyle{backmatter}

% Espaço adicional no sumário antes das referências / índice remissivo
\addtocontents{toc}{\vspace{2\baselineskip plus .5\baselineskip minus .5\baselineskip}}

% A bibliografia é obrigatória

\printbibliography[
	title=\refname\label{sec:bib}, % "Referências", recomendado pela ABNT
	%title=\bibname\label{sec:bib}, % "Bibliografia"
	heading=bibintoc, ]% Inclui a bibliografia no sumário

% \printindex % imprime o índice remissivo no documento (opcional)

\end{document}
